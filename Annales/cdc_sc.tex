% Cahier des charges - Système central
% Version 1.1

% Historique des versions
% 27/01/08 1.0 Version initiale
% 03/02/08 1.1 Version relue et corrigée

\documentclass[a4paper, 11pt, final]{article}

\usepackage[utf8]{inputenc} % Texte en utf-8
\usepackage{aeguill} % Coupure des mots accentués
\usepackage[francais]{babel} % Typographie française
\usepackage[pdftex, hypertexnames=false, colorlinks=true, final]{hyperref}
\usepackage[final]{graphicx}
\usepackage{url} % Gestion des URLs
\usepackage{placeins} % Meilleur placement des images et tableaux
\usepackage{geometry}
\usepackage{fancyhdr}
\usepackage[Lenny]{fncychap}
\usepackage{nicefrac}% pour le caractère 1/2

% Marges à gauche et à droite de 3cm
\geometry{margin=3cm}

% Utilisation des headers et footers personnalisés de fancyhdr
\pagestyle{fancy}

% Images dans le dossier ./images/
\graphicspath{{./images/}}

% Gestion des métadonnées étranges à rendre visibles au rendu
\newcommand\docname{CDCSCv1.1}
\newcommand\docauthor{Rémi Thévenoux}
\newcommand\docstatus{LIVRABLE} % EN COURS, ATTENTE, VALIDE ou LIVRABLE

% Format de citation de références standard, marche avec quasiment tout
\newcommand\fullref[1]{\ref{#1}, page \pageref{#1}}

% En-têtes et pieds de page
\lhead{\docname}
\rhead{}
\lfoot{Auteur : H4213}
\cfoot{}
\rfoot{\thepage}

% Titre du document maître
\title{\textbf{COPEVUE}\\
\rule{\textwidth}{1pt}{}\\
\Huge{\textsc{Cahier des charges\\Logiciel du système central}}}
\author{\docauthor{} (H4213)}


\date{\docname{} --- \today{} (\docstatus{})}

\begin{document}

\maketitle

\tableofcontents

\pagebreak

\section{Introduction}

\section{Présentation du projet}

\subsection{Rappel du contexte}
Il existe aujourd'hui de nombreux sites isolés et/ou difficiles d'accès qui nécessitent
une surveillance et parfois des actions à distance. Ces sites se situent dans des espaces
très différents tels que les citernes placées dans les forêts escarpées du pourtour
méditerranéen, les réservoirs utilisés pour l'autonomie des chantiers dans le grand Nord
mais aussi les personnes âgées qui se retrouvent souvent isolées.

Actuellement tous les contrôles et actions sont réalisés par un opérateur qui doit
se déplacer sur le site. Il n'y a donc que très peu de réactivité, on ne peut pas
avoir un suivi fin des évolutions et des problèmes graves -- par exemple la fuite
d'un réservoir -- ne peuvent pas être traités rapidement.

\paragraph{Étude COPEVUE}
L'objet de l'étude est la mise en place d'un système générique de surveillance
et d'action à distance sur des sites isolés. Le système devra être évolutif, autonome et fiable.

\subsection{Présentation du document}
Ce document présente le cahier des charges du logiciel du système central.
Le système central est décomposé en différentes couches, le logiciel présenté dans
le document est la couche la plus basse qui permet d'envoyer et de se connecter aux
sites pour récupérer les données, paramétrer le site et utiliser les actionneurs.

Dans le cadre du logiciel à développer, le cahier des charges sert de base à la
 rédaction des clauses contractuelles techniques, de qualité et de réception,
à partir desquelles le réalisateur proposera les spécifications fonctionnelles
et non fonctionnelles du futur logiciel. Ce sont les besoins logiciels.

Nous exprimerons ces besoins en termes d'obligation de résultats, pas d'exigence de moyens.
Ce document situe l'importance des fonctions du produit à développer pour l'application
destinée au système central et donne leurs critères d'appréciation.

\subsection{Documents applicables et de référence}
\paragraph{Documents applicables}
\begin{itemize}
\item Dossier de gestion de la documentation
\item Dossier de spécification technique des besoins
\item Dossier de faisabilité
\end{itemize}

\paragraph{Documents de référence}
\begin{itemize}
\item Plan de référence d'un cahier des charges
\end{itemize}

\section{Présentation des besoins du logiciel}
% Expression des besoins à satisfaire, but, principe du logiciel
% besoins, exploitation et ergonomie, expérience
% portée, développement, mise en oeuvre, organisation de la maintenance
% limites

\subsection{Objectifs du logiciel}
% à quel niveau intervient notre logiciel
% qui s'en sert et pour quel intérêt
% principe du logiciel, pourquoi + ergonomie

Ce logiciel sera installé sur le système central. Il est autonome et ne nécessite aucune
intervention humaine pendant sa phase de fonctionnement normal.
Son rôle principal est de récupérer les données issues des sites isolés et de les stocker.
Ce logiciel fournit ainsi des services utilisables par d'autres applications pour qu'elles
puissent lire des capteurs et utiliser des actionneurs automatiquement.
Ces applications sont spécifiques au cas d'utilisation de l'ensemble du système.
Ainsi, dans le cas des réservoirs en Norvège, on prévoira une application permettant
de gérer des itinéraires pour les intervenants.

\subsubsection{Collecter les données}
Afin de récupérer les données sur les sites isolés, la première tâche du logiciel est de régulièrement
réveiller les sites isolés pour qu'ils communiquent les valeurs de leurs capteurs, l'état du site,
l'historique des opérations et toutes les autres informations nécessaires.
Le logiciel stockera alors les données dans des fichiers XML en suivant la description de
fichiers donnée par les DTD. Un fichier XML décrira quelles sont les données à récupérer sur chaque
site et quel fichier DTD doit être utilisé pour les stocker.

\subsubsection{Gérer l'authentification}
Le logiciel devra fournir un service d'authentification afin de sécuriser l'ensemble du processus.


\subsubsection{Fournir les services}
Le logiciel devra fournir les services permettant aux applicatifs spécifiques d'interagir
avec les sites distants.

%Les services à fournir sont les suivants :
%\begin{description}
%\item [long lecture(siteId leSiteIsolé, capteurId leCapteur)] :
%Service qui permet de lire la valeur d'un capteur situé sur un site isolé.
%La valeur du capteur sera transféré avec un long (8 octets).
%\item [int actionner(siteId leSiteIsolé, actionneurId lActionneur, long laConsigne)] : 
%Ce service permet d'utiliser un actionneur, la consigne assigné à l'actionneur sera
%codé sur un long (8 octets).Le service retourne un code d'erreur sur un entier (4 octets).
%\item [int configurer(siteId leSiteIsolé, fichier leFichierdeConfig)] : 
%Ce service permet la configuration d'un site isolé. Pour cela il prend en paramètre un fichier XML
%de configuration qui décrit l'ensemble des valeurs et paramètre à configurer sur le site isolé.
%Le service retourne un code d'erreur sur un entier (4 octets).
%\end{description}

\subsection{Contraintes liées au contexte}

\subsubsection{Consultation des données}
Les données récupérées des sites isolés devront être consultables dans le temps par les différents logiciels
du serveur mais elles pourront aussi être analysées par d'autres processus. Il faut donc assurer la
pérennité des données ainsi qu'un stockage sous un format brut facilement accessible à d'autres processus.

\subsubsection{Sécurité des échanges}
Le transit des données entre les différents systèmes doit être sécurisé afin d'éviter des intrusions extérieures
sur le système. Les moyens de communication, et notamment les protocoles, devront être choisis afin de satisfaire
ce besoin.

\section{Exigences fonctionnelles}
% fonctions, performances, facteurs de qualité

\subsection{Établir une connexion SSH avec un site isolé}
Le logiciel doit être capable de connecter le serveur central au site isolé \emph{via}
une connexion SSH. Pour cela, il devra premièrement réveiller le site distant si nécessaire,
puis établir un connexion avec celui-ci.

\subsection{Récupération automatique des données}
Le logiciel doit pouvoir récupérer automatiquement les données nécessaires sur
les sites distants à intervalles réguliers.
Les données à récupérer ainsi que la périodicité du téléchargement doivent être paramétrables.

\subsection{Stockage des données}
Un fois que le logiciel a récupéré les données d'un site distant, il doit pouvoir mettre
à jour les fichiers XML de sauvegarde stockant les données. Le contenu de ces fichiers
devra être paramétrable~-- nombre de sites stockés dans le même fichier, couverture temporelle
d'un fichier, etc.~-- ainsi que la structure qui sera stockée dans un schéma au format DTD.

\subsection{Gérer l'authentification des intervenants}
Le logiciel devra fournir des services d'authentification aux intervenants et aux
applications utilisant cette couche. Il inclura une gestion des identifiants et des
mots de passe et sauvegardera les traces d'identifications. 

\subsection{Fournir les services d'interaction avec le site isolé}
\subsubsection{Service de lecture d'une donnée d'un site isolé}
Ce service prendra en paramètres l'identifiant d'un site isolé ainsi que l'identifiant
d'un capteur et retournera la valeur du capteur.
\subsubsection{Service d'utilisation d'un actionneur d'un site isolé}
Ce service prendra en paramètres les identifiants d'un site isolé et d'un actionneur,
ainsi qu'une consigne et appliquera la consigne sur l'actionneur.
\subsubsection{Service de configuration d'un site isolé}
Ce service a pour objectif de permettre la configuration d'un site isolé en fonction des données
présentes dans un fichier de configuration au format XML. Il prendra donc en paramètres l'identifiant
d'un site isolé ainsi qu'un fichier XML de configuration et configurera le site isolé en utilisant une
connexion \emph{via} SSH.

\section{Exigences non fonctionnelles}
% sûreté, faisabilité, planning, organisation, interfaces
% == contraintes non fonctionnelles

\subsection{Configurabilité}
L'ensemble des opérations doivent pouvoir être configurables, notamment par un poste distant.
Pour cela il faudra stocker les paramétrages dans des fichiers directement modifiables par un
acteurs se connectant avec le protocole SSH.

\subsection{Interopérabilité des données}%pouah j'achète du terrain là un peu.
Les données stockées sur le serveur devront être stockées dans un format le
plus universel possible, qui maximise les compatibilités et permet une interopérabilité.
De plus, le format retenu devra permettre des extensions et des traitements
potentiellement complexes des données.

\subsection{Pérennité des données}
Il faut s'assurer que des moyens soient mis en place afin de pouvoir garantir l'accès aux
données pendant plusieurs années. Notamment, se prémunir contre les risques de défaillance
du matériel de stockage, ou des catastrophes pouvant affecter le site de stockage.

\subsection{Sécurité}
Les risques d'interception des communications ou d'intrusion dans le système sont relativement faibles,
cependant nous devons assurer une certaine robustesse afin d'éviter les attaques les plus courantes.
Pour cela, le logiciel devra permettre un haut niveau de sécurité, notamment lors des communications
avec les systèmes extérieurs grâce à l'utilisation de protocoles cryptés fiables et une authentification centralisé.

\section{Analyse des exigences}
% tableaux avec la liste des exigences confrontées à l'importance et à la faisabilité

\subsection{Critères d'analyse}
Chacune des exigences fonctionnelles et non fonctionnelles est analysée et évaluée selon deux critères : sa nécessité et sa difficulté d'implémentation. Il lui est attribué deux notes sur 10, 0 signifiant par exemple la faisabilité la plus basse et 10 l'importance la plus haute.

\subsection{Analyse des exigences fonctionnelles}
\paragraph{}
\begin{figure}[h!]
\begin{center}
\begin{tabular}{|r|c|c|}
\hline
Fonction & Nécessité & Difficulté d'implémentation\\ \hline \hline
Établir une connexion SSH avec un site isolé & 8 & 6\\ \hline
Récupération automatique des données & 8 & 5\\ \hline
Stockage des données & 8 & 7\\ \hline
Gérer l'authentification des intervenants & 6 & 7\\ \hline
Service de lecture d'une donnée d'un site isolé & 5 & 4\\ \hline
Service d'utilisation d'un actionneur d'un site isolé & 5 & 4\\ \hline
Service de configuration d'un site isolé & 6 & 6\\ \hline
\end{tabular}
\end{center}
\caption{Grille d'analyse des exigences fonctionnelles}
\end{figure}
\FloatBarrier

\subsection{Analyse des exigences non fonctionnelles}
\paragraph{}
\begin{figure}[h!]
\begin{center}
\begin{tabular}{|r|c|c|}
\hline
Exigence & Nécessité & Difficulté d'implémentation\\ \hline \hline
Configuration & 8 & 6\\ \hline
Interopérabilité des données & 5 & 3\\ \hline
Pérennité des données & 7 & 4\\ \hline
Sécurité & 6 & 8\\ \hline
\end{tabular}
\end{center}
\caption{Grille d'analyse des exigences non fonctionnelles}
\end{figure}
\FloatBarrier


\section{Mise en œuvre}

\subsection{Configuration cible}
Le logiciel fonctionnera sur un serveur avec un système d'exploitation de type UNIX, certainement FreeBSD.
Il possédera un processeur d'au moins 1 Ghz ainsi que 1Go de mémoire RAM.

\subsection{Planning}

Le planning de développement de notre application est donné ci-après. Il est prévisionnel et ne prend pas en compte les phases de déploiement et de formation des intervenants.

\begin{figure}[h!]
\begin{center}
\begin{tabular}{|p{9cm}|c|}
\hline
Phase & Durée prévisionnelle\\ \hline \hline
Développement et tests unitaires & 3 \nicefrac{1}{2} mois\\ \hline
Tests en simulation et tests d'intégration & 2 mois \\ \hline
Tests avec matériel et retour d'expériences & 1 \nicefrac{1}{2} mois\\ \hline
Finalisation, éventuellement reprise de code posant des erreurs, le retour d'expérience nous indiquant les lacunes à combler & 1 mois\\ \hline
Total & 8 mois\\ \hline
\end{tabular}
\end{center}
\caption{Planning prévisionnel de livraison du logiciel à destination du système central}
\end{figure}
\FloatBarrier

%\appendix

% \section{Étude préalable}
% étude de l'existant, contexte du logiciel dans la société du demandeur

% rien

%\section{Étude de faisabilité}
%élaboration des solutions possibles
% cf Étude de faisabilité}


\end{document}
