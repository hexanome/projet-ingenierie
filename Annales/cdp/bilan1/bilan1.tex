\documentclass[a4paper, 11pt, draft]{article}

\usepackage[utf8]{inputenc} % Texte en utf-8
\usepackage{aeguill}
\usepackage[francais]{babel} % Typographie française

% Titre du document maître
\title{\textbf{COPEVUE}\\
\rule{\textwidth}{1pt}{}\\
\Huge{\textsc{Bilan chef de projet}}}
\author{Guillaume Bouchard}
\date{\today{}}

\begin{document}

\maketitle

Ce document présente le bilan du chef de projet sur le projet de gestion de
site distant. Il mélange deux visions du projet. La première concerne la
vision de l'ingénieur en temps que chef de projet, et la seconde est mon
bilan personnel en temps qu'élève ingénieur dans cette expérience de chef de
projet.

\paragraph{}

Dans la globalité, le résultat de ce projet est une réussite, nous avons su
dégager une solution qui tient la route et qui a même été qualifié
d'innovante, ce qui est gratifiant.

\paragraph{}

Concernant la qualité du travail fournie par l'équipe, elle est très
satisfaisante. J'ai seulement un petit regret concernant le dossier
d'interface et un des cahiers des charges qui à mon avis ont été bâclés sur
la fin.

\paragraph{}

Il fut très difficile de savoir jusqu'à quel point pousser certains
documents.  Je prend l'exemple des cahiers des charges qui ne sont à mon
sens pas complets sans certaines descriptions très précises -- structures
des fichiers XML échangés, informations sur les interfaces graphiques\ldots
--.

Mais le temps impartit n'était pas suffisant, il aurait fallu plusieurs
semaines à temps plein, ce qui impose ces imprécisions dans certains de ces
rendus.


\paragraph{}

J'ai eu de nombreux problèmes pour me faire respecter en temps que chef de
projet et pour imposer du travail à mes collaborateurs.

Une des première raison à cela est tout bonnement qu'il est difficile de
donner des ordres et de critiquer des amis proche. Autant il s'agit de
personnes avec qui j'aime travailler, autant je pense qu'il aurait été plus
simple d'effectuer ce projet avec des inconnus.

\paragraph{}

Je regrette le manque d'implication de groupe au sein de mon équipe, bien
souvent personne ne prenait connaissance du travail des autres membres et
j'ai eu le regret de découvrir la dernière semaine du projet que plusieurs
personnes n'avaient pas consultés le dossier de conception ainsi que le
dossier de qualité, ce qui impliqua des cahiers des charges non cohérents
avec la conception initiale et des livrables ne suivant pas les consignes de
qualité.

\paragraph{}

Un des plus grand problème au niveau de ce projet et que toute l'équipe a
rencontré concerne la capacité à respecter des échéance. Tous les membres de
cet équipe travaillent très bien et avec bonne volonté, mais il doivent se
sentir en situation de stress pour commencer à travailler, ce qui arrive
bien souvent quelques jours après la deadline.

\paragraph{}

J'ai pour ma part eu beaucoup de mal à appréhender un travail non technique
et j'ai bien souvent eu l'impression de faire du recopiage stérile sans
valeur ajoutée.

Lors de mon travail je me suis inspiré de beaucoup de documents trouvés
autant sur internet que dans les annales de ce projet et bien souvent
j'étais tenté de recopier mot pour mot ce qui y était écrit, tant cela
correspondait, dans un meilleur style, à ce que j'aurais écris.

C'est d'ailleurs une des raisons de la taille réduite du PMP que j'ai rendu,
je me suis contenté de mettre dedans que ce que j'estime apporter une valeur
ajoutée.

\paragraph{}

En conclusion, je retiendrais un projet intéressant et quels sont les
erreurs, ou au moins une partie de celle-ci, à ne pas commettre en temps que
chef de projet. L'erreur principale étant de laisser trop de liberté au GEI.

\end{document}
