% Annexes au dossier de qualité
% Version 1.2

% Historique des versions
% 20/01/08 1.0 Création et remplissage (Mikado)
% 21/01/08 1.1 Plans types : STB et DCNS (Mikado)
% 23/01/08 1.2 Plan type : CDC (Mikado)

% \renewcommand\thechapter{\alph{chapter}}
\renewcommand\docname{ANNEXESv1.1}
\appendix

\chapter{Modèles de documents}

\label{chapter:modeles_documents}

L'uniformisation des documents est obtenue en utilisant des templates \LaTeX{} définis par le responsable qualité, en accord avec le reste du groupe.

\section{Modèle de document livrable}

\paragraph{}
Un document livrable doit présenter les informations suivantes :
\begin{itemize}
\item En page de garde :
\begin{itemize}
\item Le titre du projet
\item Le titre du document
\item L'auteur principal du document
\item L'auteur (hexanôme) du document
\item L'identification du document
\item La date de compilation telle qu'obtenue par la commande \verb|\today|
\item L'état du document
\end{itemize}
\item En en-tête :
\begin{itemize}
\item L'identification du document courant
\end{itemize}
\item En bas de page :
\begin{itemize}
\item L'auteur (hexanôme) du document
\item Le numéro de la page courante dans le document
\end{itemize}
\end{itemize}
Certaines informations ne seront accessibles que via la lecture de la source du document, par exemple l'historique des versions (qui permet d'obtenir la dernière personne à avoir contribué au document, sa contribution, les dates de création et de modification\ldots).

Il faut également noter que pour respecter au maximum les règles de la bonne typographie, certaines pages (nouveaux chapitres) ne présentent pas toutes les informations dans les en-têtes et pieds de page.

\paragraph{}
Un document livrable doit commencer par le chapitre d'introduction, qui comporte les sections suivantes : 

(Dans le corps de l'introduction : identification, auteur et état du document)
\begin{enumerate}
\item Présentation du projet
\item Présentation du document
\begin{enumerate}
\item Objectifs
\end{enumerate}
\item{Documents applicables et de référence}
\begin{enumerate}
\item Documents applicables
\item Documents de référence
\end{enumerate}
\end{enumerate}

\chapter{Plans types}

\label{chapter:plans_types}

\section[STB]{Dossier de spécification technique des besoins}

\begin{enumerate}
\item{Introduction}
\item{Axes d'amélioration retenus}
\begin{enumerate}
\item{Axes de progrès retenus}
\item{Axes de progrès marginaux}
\item{Faux axes de progrès éventuels}
\end{enumerate}
\item{Description des exigences fonctionnelles du futur système}
\item{Description des exigences non fonctionnelles du futur système}
\item{Impacts de la nouvelle solution sur le système}
\item{Bilan des améliorations}
\item{Conclusion}
\end{enumerate}

\section[CNS]{Dossier de conception de nouveau système}

\begin{enumerate}
\item Introduction
\item Organisation générale de l'atelier
\item Règles de pilotage de l'atelier
\item Architecture applicative
\item Architecture informatique et matérielle
\item Réflexions sur les données
\item Gestion des anomalies et sécurité
\item Conclusion
\item[A]{Annexe : Représentation informatique des objets technologiques}
\item[B]{Annexe : Réflexion sur le réseau}
\item[C]{Annexe : Démarrage du système}
\end{enumerate}

\section[CDC]{Cahier des charges}

\begin{enumerate}
\item Introduction
\item Objectif du logiciel
\item Exigences fonctionnelles
\item Contraintes imposées et faisabilité technologique
\item Configuration cible
\item[A]{Annexe : Etude préalable}
\item[B]{Annexe : Etude de faisabilité}
\end{enumerate}
