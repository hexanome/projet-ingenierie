% Mikado : les commentaires proviennent du dossier Documentation du RQ

\section{Introduction}

\section{Objectif du logiciel}
% Expression des besoisns à satisfaire, but, principe du logiciel
% besoins, exploitation et ergonomie, expérience
% portée, développement, mise en oeuvre, organisation de la maintenance
% limites

\section{Exigences fonctionnelles}
% fonctions, performances, facteurs de qualité

\section{Contraintes imposées et faisabilité technologique}
% sûreté, faisabilité, planning, organisation, interfaces

\section{Configuration cible}
% matériels et logiciels à utiliser

\appendix

\section{Etude préalable}
% étude de l'existant, contexte du logiciel dans la société du demandeur

\section{Etude de faisabilité}
% élaboration des solutions possibles
