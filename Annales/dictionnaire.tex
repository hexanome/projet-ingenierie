\documentclass[a4paper]{report}

\usepackage[utf8]{inputenc} % Texte en utf-8
\usepackage[cyr]{aeguill} % Permet la coupure des mots accentués
\usepackage[francais]{babel} % Typographie française
\usepackage[pdftex,colorlinks=true,hypertexnames=false]{hyperref} % Liens hypertextes

\author{H4213}
\title{Copevue\\Dictionnaire des termes utilisés}

\begin{document}
\maketitle

\subsubsection{Alimentation énergétique des sites isolés}
Batterie et systèmes d'alimentation facultative.

\subsubsection{Intervention}
Action d'un intervenant sur un site isolé. Peut être une action de
maintenance préventive ou corrective, de maintenance de la
fonctionnalité du site, etc.

\subsubsection{Intervenant}
Personne effectuant des opérations sur les sites isolés.

\subsubsection{Poste de gestion}
Poste permettant de gérer le système central.

\subsubsection{Site isolé}
Site où sont effectuées les mesures.

\subsubsection{Système central}
Site ou sont stockées toutes les données. L'archivage des
planifications y est également effectué.

\subsubsection{Système de l'intervenant}
Système permettant à l'intervenant de recevoir et d'envoyer des informations pendant ces interventions.

\subsubsection{Système de communication}
Tout système utilisé permettant de faire communiquer les sites isolés,
le système central et les systèmes des intervenants. Dans notre cas,
ce sont le GSM et Internet.

\end{document}
