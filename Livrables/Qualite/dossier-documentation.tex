% instanciation de la gestion de la documentation
% droit de tenter des expériences pour l'organisation de la production (auto-critique)
% Annexes

\documentclass[a4paper]{article}

\usepackage[utf8]{inputenc}   
\usepackage[top=2cm, bottom=2cm, left=2cm, right=2cm]{geometry}
\usepackage{ucs}
% Reconnaitre les caratères accentués dans le source.
\usepackage[T1]{fontenc} 
\usepackage{lmodern}
\usepackage[francais]{babel}
% Insertion d'images
\usepackage{graphicx}


\begin{document}

\part{Introduction}

\section{Rappel}

\paragraph{Documentation} Ensemble de documents relatifs à un projet - notice - mode d’emploi - action de sélectionner, classer, utiliser ou diffuser des documents.

La documentation d’un projet a une importance primordiale : c’est l’outil de communication et de dialogue entre les membres de l’équipe projet et les intervenants extérieurs (membres des comités de pilotage et utilisateurs, chef de projet, coordinateurs des projets, utilisateurs, etc...). Elle assure aussi la pérennité des informations au sein du projet.

Afin d’organiser la gestion de la documentation produite par projet, il convient au préalable d’identifier tous les types de documents relatifs aux diverses étapes d’un projet, de les référencer de manière homogène pour ensuite définir un mode de gestion commun à tous les projets.

\section{Objet du document}

Le dossier de gestion et d'organisation de la documentation du projet a pour objectif de définir l’ensemble des règles
communes concernant la gestion de la documentation (structuration, page de garde, cycle de vie d’un
document, gestion de version, structuration du système documentaire, sauvegardes et diffusion, ...) et
l’organisation de la documentation en définissant des plans types (voir en annexe) pour les documents relatifs à la gestion de
projet ainsi que les documents relatifs à la qualité et les annexes relatives aux études.

Nous proposerons des règles de gestion de la documentation pour ce projet qui permettront de mettre en oeuvre des moyens
de référenciation homogènes de l'ensemble de la documentation relative au projet, d'en organiser la production, le classement
et l'accès.

\part{Gestion de la documentation}

Ce chapitre précise les règles de gestion de la documentation à mettre en oeuvre dans tout projet.

Pour mieux comprendre la nécessité d’une gestion rigoureuse de la documentation, il convient en premier lieu de détailler les états par lesquels passe un document avant d’être diffusé ainsi que le rôle des différents acteurs.

\section{Les acteurs et leurs responsabilités}

Les différents acteurs sont :

\begin{itemize}
\item le chargé de la gestion documentaire (généralement le responsable qualité du projet),
\item le(s) auteur(s) du document,
\item les responsables de la vérification (membres de l’équipe projet ou intervenants extérieurs),
\item les responsables de la validation (une ou plusieurs personnes désignées).
\end{itemize}

% tableau

\section{Cycle de vie d'un document}

Un document passe ou peut passer par un certain nombre d'états :

\begin{itemize}
\item travail : le document est en cours d'élaboration par l'auteur
\item terminé : le document satisfait l'auteur; il est prêt à être diffusé
\item vérifié (optionnel) : le document est approuvé par d'autres membres de l'équipe, des intervenants extérieurs et/ou le contrôle qualité
\item validé : le document est approuvé par les personnes habilitées et prend valeur de référence au sein du projet
\item périmé : le document n’est plus adapté et est donc retiré à tous ses détenteurs (retrait d'usage)
\item archivage : le document n'est plus consulté régulièrement, mais une trace de son existence demeure (pour une durée définie par le chargé de gestion de la documentation du projet)
\item destruction : le document n'est pas archivé ou le délai d'archivage est écoulé
\end{itemize}

% schema

\subsection{Production du document}

Un document en cours de production est dans l'état ``travail''.

Lorsque l'auteur obtient une rédaction qui le satisfait et ne souhaite plus apporter de modifications, il l'indique en le faisant passer à l'état ``terminé''.
Avant de faire passer un document en l'état ``terminé'', l'auteur peut le soumettre à des lectures croisées au sein de son équipe.

\subsection{Vérification / validation du document}

L'auteur diffuse alors le document aux vérificateurs puis aux validateurs, ou directement aux validateurs (la vérification est optionnelle selon le type de document). La diffusion se fait sous format papier ou électronique (choisir le plus pratique).

Il joint à son document une fiche de relecture où les remarques éventuelles des vérificateurs ou validateurs sont formalisées (modifications souhaitées).

Toutes les remarques de fond sur le contenu du document (imprécisions, ambiguïtés, incohérences...) doivent être consignées dans cette fiche sauf les remarques relatives à la forme du document (fautes de frappe, d'orthographe, problèmes de mise en page...) qui peuvent être signalées directement sur la copie papier du document.

Si les modifications du texte sont importantes, elles sont juste référencées dans la fiche de relecture puis décrites directement sur une copie papier du document.

La fiche de relecture comporte les éléments suivants : 

\paragraph{Une partie renseignée par l'auteur (avant transmission au vérificateur / validateur)}

\begin{itemize}
\item nom du demandeur
\item date de la demande
\item nom et référence du document
\item date de retour pour les remarques
\item aspects à examiner (contenu, forme, totalité, partie...)
\end{itemize}

\paragraph{Une partie renseignée par le vérificateur/validateur}

\begin{itemize}
\item nom du vérificateur ou validateur
\item date de vérification ou validation
\item conclusion de la vérification ou validation :
\item document validé,
\item document validé après intégration des modifications par l'auteur,
\item document à revalider (nécessite un nouveau passage en vérification/validation après intégration des modifications par l'auteur),
\item liste des points à modifier dans le document (numéro de §, page, description de la modification ou référence à une annotation dans la copie papier du document jointe).
\end{itemize}

Cette fiche (ainsi qu'éventuellement le document annoté joint) est transmise à l'auteur.
L'auteur répond aux remarques émises par les relecteurs dans la colonne ``justification réponses'' de la fiche prévue à cet effet.
L'auteur conserve une copie papier de la fiche.

Si la vérification / validation est acceptée, le document passe à l'état ``vérifié'' / ``validé'', sinon il revient en état de ``travail''.
L'auteur du document est chargé d'indiquer en page de garde du document l'état dans lequel le document se trouve, ainsi que les noms des vérificateurs / validateurs et les dates de vérification / validation.

\paragraph{NB} Pour chaque document à valider, une date de retour des remarques est convenue. Si aucun retour n'est parvenu à l'auteur à la date prévue, le document est considéré comme validé.

\subsection{Archivage du document}

Lorsqu'un document est périmé, le responsable chargé de la gestion de la documentation l'archive et veille à informer tous les détenteurs du document de sa cessation d'applicabilité.

\section{Identification et structure de la documentation}

\subsection{Identification}

Afin d’assurer l’efficacité de la gestion de la documentation, il faut prévoir un mécanisme normalisé d’identification des documents (homogénéité).

Ainsi, chaque document reçoit une référence unique au sein du projet.

\subsection{Structure}

\section{Gestion des versions - révisions}

Chaque modification d’un document doit être faite en accord avec les dispositions d’approbation des documents (voir chapitre : "cycle de vie d’un document"). 

L’auteur de la modification est responsable :

\begin{itemize}
\item du respect du cycle de vérification et validation,
\item de la diffusion des nouvelles versions,
\item de la révision des modifications.
\end{itemize}

L'indice de révision est propre à une modification :

\begin{itemize}
\item il est incrémenté à chaque modification de contenu ou de forme sur un ou plusieurs documents.
\end{itemize}

Pour certains documents, il est important de faire apparaître clairement les évolutions d'une révision à l'autre du document. A cet effet, la procédure de ``commit'' du gestionnaire de révisions est utilisée de la façon suivante :

\begin{itemize}
\item sauvegarder et fermer le document
\item entammer la procédure de ``commit'' du gestionnaire de versions ``Git''
\item dans la partie ``Message de commit'', indiquer sous forme de liste à puces les changements incrémentaux qui ont été apportés sur le document
\item valider le commit
\item puis deux possibilités :

\begin{itemize}
\item effectuer les modifications en une ou plusieurs séances de travail, puis faire apparaître les modifications par rapport à l'ancienne version :

\begin{itemize}
\item valider le dernier commit
\item synchroniser le répertoire de travail avec le serveur centralisé sur GitHub
\end{itemize}

\item faire apparaître directement les modifications :

\begin{itemize}
\item synchroniser directement avec le répertoire de travail central après votre premier commit
\end{itemize}

\end{itemize}
\end{itemize}

GitHub affiche la version modifiée du document (exemple : dans dossier-documentation.tex) avec des marques de révision indiquant le texte corrigé (en rouge le texte qui a été supprimé, en vert celui qui a été ajouté). Ces marques de révision resteront actives dans le document et consultables à tout moment.

\paragraph{Remarques}

\begin{itemize}
\item si une image est modifiée, une marque de révision simplifiée apparaît dans la vue de consultation de l'historique sur GitHub
\item les corrections de fautes d'orthographe, de frappe ou de mise en page ne doivent pas apparaître avec des marques de révision. Pour cela, effectuer ce type de modifications sur le document initial (exemple : dossier-documentation.tex).
\end{itemize}

\paragraph{Attention} Le document original choisi pour effectuer la révision (dans notre exemple dossier-documentation.tex), est fonction du destinataire auquel s'adresse cette révision :

\begin{itemize}
\item soit le destinataire aura en main toutes les révisions successives du document ;
\item soit le document transmis fera la somme de toutes les relectures successives.
\end{itemize}


\section{Outils de production de la documentation}

Tout nouveau document est produit sur PC ou Macintosh, en utilisant un logiciel d'édition de document compatible avec la technologie Latex, permettant de générer des documents homogènes et de grande qualité.

\section{Gestion physique des fichiers contenant les documents}

\subsection{Répertoires}

Les documents du projet sont accessibles en consultation à tous les membres de l'équipe projet sur le serveur ou sur leur propre copie du répertoire de travail, dans le dossier ``Livrables''. Un sous-répertoire est créé par nature de document à gérer, comme par exemple : ``Qualité'', ``Dossier\_Initialisation'', etc.

\subsection{Noms des fichiers}

Le nom de chaque fichier représentant un livrable est unique, et s'écrit avec les mots du nom du livrable, ou un sous-ensemble de mots-clefs représentatifs, séparés par des tirets.

L'extension ``.tex'' est rajoutée systématiquement afin de pouvoir facilement accéder au document depuis un PC ou un MAC.

\paragraph{Exemples} ``dossier-documentation.tex'', ``glossaire-commun.tex'', etc.

\subsection{Procédures de sauvegarde et archivage}

Le répertoire du projet est sauvegardé en continu par la procédure automatique des serveurs de GitHub. Les différentes versions des documents sont archivées par le gestionnaire de révision Git, qui référence les sources Latex de tous les documents et enregistre leurs modifications au cours du temps, au sein du répertoire de travail Git.

La copie centrale de ce répertoire de travail se trouve sur les serveurs de GitHub, et chaque membre travaillant sur les documents possède sa propre copie du répertoire de travail, laquelle il synchronise régulièrement avec le système de révision central. Ainsi, en cas de panne majeur sur la copie centrale ou sur l'une des copies, aucun travail n'est perdu car il suffit de restaurer un répertoire de travail en le clonant à partir d'un autre.

Lorsque l'accès immédiat à un document n'est plus indispensable, le document est archivé par sécurité. Si le document est périmé, le responsable de la documentation veille à informer tous les détenteurs du document de sa cessation d'applicabilité.

\part{Organisation de la production}

\paragraph{Rappel : organisation d'un projet} L'équipe projet, dirigée par le Chef de Projet (CdP) est composée de plusieurs GEI ainsi que d'un Responsable Qualité (RQ).

Voici une liste rapide des documents à rendre pour chaque groupe de travail (livrables de la phase 1 pour le CdP et le GEI). 
Pour plus de détails concernant ces documents, vous pouvez vous référer aux informations contenues dans le dossier d'initialisation.

\section{Documents de gestion de projet}

\begin{itemize}
\item Dossier d'Initialisation
\item Relevé de Décisions
\item Fiche d'Argumentation Commerciale 
\item Rédaction d'une procédure : Le chef de projet devra rédiger la procédure de son choix à savoir : soit la \og décomposition d'un sytème et choix de sous projets \fg ou \og gestion de configuration \fg.
\item Best practise 1 : Rédaction d'une procédure
\item Best practise 2 : Rédaction d'un cahier des charges logiciel
\end{itemize}


\section{Documents d’étude et développement}

\begin{itemize}
\item Dossier de faisabilité
\item Dossier de spécification technique des besoins
\item Ebauche de la conception détaillée
\item Glossaire commun
\item Dossier de synthèse
\end{itemize}

\section{Documents relatifs à la qualité}

\begin{itemize}
\item Dossier de structuration de la documentation et organisation de la production
\item Plan d'Assurance Qualité Projet simplifié (niveau MOA)
\item Plan d'Assurance Qualité Projet simplifié sous-projet logiciel ``station centrale'' (Draft)
\item Critique formelle sur les documents de la première partie
\item Critique formelle du point de vue du projet
\item Dossier bilan personnel du Responsable Qualité
\end{itemize}

\part{Annexes}

\paragraph{Plans types} Pour consulter une série de plans types utiles pour la réalisation de documents, voir l'annexe ``dossier-documentation-annexe-1-plan-type.pdf''.
\paragraph{Manuel Git} Pour s'initier à l'utilisation du gestionnaire de révisions distribué Git, voir l'annexe ``dossier-documentation-annexe-2-tutoriel-git.pdf''.
\paragraph{Introduction Latex} Pour une petite introduction à la technologie de documentation Latex, voir l'annexe ``dossier-documentation-annexe-3-introduction-latex.pdf''.

\end{document}
