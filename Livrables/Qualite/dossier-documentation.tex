% instanciation de la gestion de la documentation
% droit de tenter des expériences pour l'organisation de la production (auto-critique)
% Annexes

\documentclass[a4paper]{article}

\usepackage[utf8]{inputenc}   
\usepackage[top=2cm, bottom=2cm, left=2cm, right=2cm]{geometry}
\usepackage{ucs}
% Reconnaitre les caratères accentués dans le source.
\usepackage[T1]{fontenc} 
\usepackage{lmodern}
\usepackage[francais]{babel}
% Insertion d'images
\usepackage{graphicx}


\begin{document}

\part{Introduction}

\section{Rappel}

\paragraph{Documentation} Ensemble de documents relatifs à un projet - notice - mode d’emploi - action de sélectionner, classer, utiliser ou diffuser des documents.

La documentation d’un projet a une importance primordiale : c’est l’outil de communication et de dialogue entre les membres de l’équipe projet et les intervenants extérieurs (membre des comités de pilotage et utilisateurs, chef de projet, coordination des projets, utilisateurs, etc...). Elle assure aussi la pérennité des informations au sein du projet.

Afin d’organiser la gestion de la documentation produite par projet, il convient au préalable d’identifier tous les types de documents relatifs aux diverses étapes d’un projet, de les référencer de manière homogène pour ensuite définir un mode de gestion commun à tous les projets.

\section{Objet du document}

Le dossier de gestion et d'organisation de la documentation du projet a pour objectif de définir l’ensemble des règles
communes concernant la gestion de la documentation (structuration, page de garde, cycle de vie d’un
document, gestion de version, structuration du système documentaire, sauvegardes et diffusion, ...) et
l’organisation de la documentation en définissant des plans types (voir en annexe) pour les documents relatifs à la gestion de
projet, les documents relatifs à la qualité et les annexes relatives aux études.

Nous proposerons des règles de gestion de la documentation pour ce projet qui permettront de mettre en oeuvre des moyens
de référenciation homogène de l'ensemble de la documentation relative au projet, d'en organiser la production, le classement
et l'accès.

\part{Gestion de la documentation}

Ce chapitre précise les règles de gestion de la documentation à mettre en oeuvre dans tout projet.

Pour mieux comprendre la nécessité d’une gestion rigoureuse de la documentation, il convient en premier lieu de détailler les états par lesquels passe un document avant d’être diffusé ainsi que le rôle des différents acteurs.

\section{Les acteurs et leurs responsabilités}

Les différents acteurs sont :

\begin{itemize}
\item le chargé de la gestion documentaire (généralement le responsable qualité du projet),
\item le(s) auteur(s) du document,
\item les responsables de la vérification (membres de l’équipe projet ou intervenants extérieurs),
\item les responsables de la validation (une ou plusieurs personnes désignées).
\end{itemize}

% tableau

\section{Cycle de vie d'un document}

Un document passe ou peut passer par un certain nombre d'états :

\begin{itemize}
\item travail : le document est en cours d'élaboration par l'auteur
\item terminé : le document satisfait l'auteur; il est prêt à être diffusé
\item vérifié (optionnel) : le document est approuvé par d'autres membres de l'équipe, des intervenants extérieurs et/ou le contrôle qualité
\item validé : le document est approuvé par les personnes habilitées et prend valeur de référence au sein du projet
\item périmé : le document n’est plus adapté et est donc retiré à tous ses détenteurs (retrait d'usage)
\item archivage : le document n'est plus consulté régulièrement, mais une trace de son existence demeure (pour une durée définie par le chargé de gestion de la documentation du projet)
\item destruction : le document n'est pas archivé ou le délai d'archivage est écoulé
\end{itemize}

% schema

\subsection{Production du document}

Un document en cours de production est dans l'état ``travail''.

Lorsque l'auteur obtient une rédaction qui le satisfait et ne souhaite plus apporter de modifications, il l'indique en le faisant passer à l'état ``terminé''.
Avant de faire passer un document en l'état ``terminé'', l'auteur peut le soumettre à des lectures croisées au sein de son équipe.

\subsection{Vérification / validation du document}

L'auteur diffuse alors le document aux vérificateurs puis aux validateurs, ou directement aux validateurs (la vérification est optionnelle selon le type de document). La diffusion se fait sous format papier ou électronique (choisir le plus pratique).

Il joint à son document une fiche de relecture où les remarques éventuelles des vérificateurs ou validateurs sont formalisées (modifications souhaitées).

Toutes les remarques de fond sur le contenu du document (imprécision, ambiguïtés, incohérences...) doivent être consignées dans cette fiche sauf les remarques relatives à la forme du document (fautes de frappe, d'orthographe, problèmes de mise en page...) qui peuvent être signalées directement sur la copie papier du document.

Si les modifications du texte sont importantes, elles sont juste référencées dans la fiche de relecture puis décrites directement sur une copie papier du document.

La fiche de relecture comporte les éléments suivants : 

\paragraph{Une partie renseignée par l'auteur (avant transmission au vérificateur / validateur)}

\begin{itemize}
\item nom du demandeur
\item date de la demande
\item nom et référence du document
\item date de retour pour les remarques
\item aspects à examiner (contenu, forme, totalité, partie...)
\end{itemize}

\paragraph{Une partie renseignée par le vérificateur/validateur}

\begin{itemize}
\item nom du vérificateur ou validateur
\item date de vérification ou validation
\item conclusion de la vérification ou validation :
\item document validé,
\item document validé après intégration des modifications par l'auteur,
\item document à revalider (nécessite un nouveau passage en vérification/validation après intégration des modifications par l'auteur),
\item liste des points à modifier dans le document (numéro de §, page, description de la modification ou référence à une annotation dans la copie papier du document jointe).
\end{itemize}

Cette fiche (ainsi qu'éventuellement le document annoté joint) est transmise à l'auteur.
L'auteur répond aux remarques émises par les relecteurs dans la colonne ``justification réponses'' de la fiche prévue à cet effet.
L'auteur conserve une copie papier de la fiche.

Si la vérification / validation est acceptée, le document passe à l'état ``vérifié'' / ``validé'', sinon il revient en état de ``travail''.
L'auteur du document est chargé d'indiquer en page de garde du document l'état dans lequel le document se trouve, ainsi que les noms des vérificateurs / validateurs et les dates de vérification / validation.

\paragraph{NB} Pour chaque document à valider, une date de retour des remarques est convenue. Si aucun retour n'est parvenu à l'auteur à la date prévue, le document est considéré comme validé.

\subsection{Archivage du document}

Lorsqu'un document est périmé, le responsable chargé de la gestion de la documentation l'archive et veille à informer tous les détenteurs du document de sa cessation d'applicabilité.

\section{Identification et structure de la documentation}

\subsection{Identification}

Afin d’assurer l’efficacité de la gestion de la documentation, il faut prévoir un mécanisme normalisé d’identification des documents (homogénéité).

Ainsi, chaque document reçoit une référence unique au sein du projet.

\subsection{Structure}

\section{Gestion des versions - révisions}

Chaque modification d’un document doit être faite en accord avec les dispositions d’approbation des documents (voir chapitre : "cycle de vie d’un document"). 

L’auteur de la modification est responsable :

\begin{itemize}
\item du respect du cycle de vérification et validation,
\item de la diffusion des nouvelles versions,
\item de la révision des modifications.
\end{itemize}

Le numéro de version qui apparaît sur un document correspond à la version de l'application logicielle concernée.

L'indice de révision est propre à une modification :

\begin{itemize}
\item il est incrémenté à chaque modification de contenu ou de forme sur un ou plusieurs documents.
\end{itemize}

Pour certains documents, il est important de faire apparaître clairement les évolutions d'une révision à l'autre du document. A cet effet, l'option révision de Word est utilisée de la façon suivante :

\begin{itemize}

\item sauvegarder le document avec son indice de révision en dernier caractère (exemple : EDASTTO2.doc  contient la révision 2 du document EDASTTO)

\item dupliquer le document à modifier et le sauvegarder avec l'indice de révision incrémenté (exemple : EDASTTO3.doc)

\item supprimer toute trace des évolutions antérieures s'il y en avait (exemple : dans EDASTTO3.doc), pour cela :

\begin{itemize}
\item choisir Outils Révisions
\item choisir "Accepter tout"
\end{itemize}

\item puis deux possibilités :

\begin{itemize}

\item effectuer les modifications en une ou plusieurs séances de travail (exemple : dans EDASTTO3.doc) puis faire apparaître les modifications par rapport à l'ancienne version :

\begin{itemize}
\item choisir Outils Révisions
\item choisir "Comparer versions"
\item sélectionner le nom du document original (exemple : EDASTTO2.DOC), puis "Ouvrir"
\end{itemize}

\item faire apparaître directement les modifications :

\begin{itemize}
\item choisir Outils Révisions
\item cocher la case "Marquer les corrections en cours de frappe"
\end{itemize}

\end{itemize}
\end{itemize}

Word affiche la version modifiée du document (exemple : dans EDASTTO3.doc) avec des marques de révision indiquant le texte corrigé (trait à gauche). Ces marques de révision resteront actives dans le document jusqu'à la prochaine évolution.

\paragraph{Remarques}

\begin{itemize}
\item pour que seules des marques de révision apparaissent dans les marges, choisir Outils Révisions Options, positionner la marque du Texte inséré à (Aucune) , la marque du Texte supprimé à Masqué et la marque des Lignes revues à Barre à gauche
\item si une image est modifiée, aucune marque de révision n'apparaît dans la marge
\item les corrections de fautes d'orthographe, de frappe ou de mise en page ne doivent pas apparaître avec des marques de révision. Pour cela, effectuer ce type de modifications sur le document initial (exemple : dans EDASTTO2.doc) avant de le dupliquer.
\end{itemize}

\paragraph{Attention} Le document original choisi pour effectuer la révision (dans notre exemple EDASTTO2.DOC), est fonction du destinataire auquel s'adresse cette révision :

\begin{itemize}
\item soit le destinataire aura en main toutes les révisions successives du document ;
\item soit le document transmis fera la somme de toutes les relectures successives.
\end{itemize}


\section{Outils de production de la documentation}

Tout nouveau document est produit sur PC ou Macintosh, en utilisant un logiciel d'édition de document compatible avec la technologie Latex, permettant de générer des documents homogènes et de grande qualité.

\section{Classement}

\section{Gestion physique des fichiers contenant les documents}

\subsection{Répertoires}
\subsection{Noms des fichiers}
\subsection{Procédures de sauvegarde et archivage}

Le répertoire du projet est sauvegardé en continu par la procédure automatique des serveurs de GitHub. Les différentes versions des documents sont archivées par le gestionnaire de révision Git, qui référence les sources Latex de tous les documents et enregistre leurs modifications au cours du temps, au sein du répertoire de travail Git.

La copie centrale de ce répertoire de travail se trouve sur les serveurs de GitHub, et chaque membre travaillant sur les documents possède sa propre copie du répertoire de travail, laquelle il synchronise régulièrement avec le système de révision central. Ainsi, en cas de panne majeur sur la copie centrale ou sur l'une des copies, aucun travail n'est perdu car il suffit de restaurer un répertoire de travail en le clonant à partir d'un autre.

Lorsque l'accès immédiat à un document n'est plus indispensable, le document est archivé par sécurité. Si le document est périmé, le responsable de la documentation veille à informer tous les détenteurs du document de sa cessation d'applicabilité.

\part{Organisation de la production}

\section{Documents de gestion de projet}
\section{Documents d’étude et développement}
\section{Documents relatifs à la mise en oeuvre}
\section{Documents relatifs à la qualité}

\part{Annexes}

\paragraph{Plans types} Voir document joint en annexe.

% plans types

\end{document}
