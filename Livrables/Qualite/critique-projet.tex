%-------------------------------------------
% En-tête type de document pour le projet PLD
% Il suffit de remplir le input ligne 45
%-------------------------------------------

\documentclass[a4paper]{article}

\usepackage[utf8]{inputenc}   
\usepackage[top=2cm, bottom=2cm, left=2cm, right=2cm]{geometry}
\usepackage{ucs}
% Reconnaitre les caratères accentués dans le source.
\usepackage[T1]{fontenc} 
\usepackage{lmodern}
\usepackage[francais]{babel}
% Insertion d'images
\usepackage{graphicx}
% Utilisation du symbole EURO
\usepackage{eurosym}
% Usage de liens.
\usepackage{hyperref}
\hypersetup{
	pdfborder={0 0 0}
}

\setlength{\parskip}{10pt plus 1pt minus 1pt}

\begin{document}

%------------------------------------- Page de titre
\begin{titlepage}
~ 
\vfill
	\begin{center}
		\begin{Huge}
		Projet D'ingénierie : Critique Formelle\\
		\end{Huge} 
\vfill
		\textbf{Hexanome 4111 :} 
		\\Quentin \bsc{Calvez}, Matthieu \bsc{Coquet}, 
		\\Jan \bsc{Keromnes}, Alexandre \bsc{Lefoulon}, 
		\\Thaddée \bsc{Tyl}, Xavier \bsc{Sauvagnat},
		\\Tuuli \bsc{Tyrväinen}
\vfill		
		\begin{Large}
		Janvier 2012
		\end{Large}
\vfill
	\begin{tabular}{|c|c|c|c|c|}
 	 \hline
   Destinataire & Version & Etat & Dernière révision & Equipe \\
   \hline
   Professeurs & 1 & Validé & \today & H4111 \\
   \hline
	\end{tabular}
	\end{center}
\vfill
\end{titlepage}
%----------------------------------------------------
%--------------------------------- Table des matières
\newpage
\tableofcontents
\newpage
%----------------------------------------------------

\section{Introduction}

L'objectif de ce document est de faire un point sur les documents produits lors de la seconde et dernière phase du projet d'ingénierie, en se focalisant sur l'aspect Qualité. Cette analyse sera également l'occasion de dresser une retrospective du projet, en identifiant ses points forts et ses points faibles. Nous verrons également si les attentes exprimées en fin de première partie ont été atteintes.

\section{Objectifs critiques}

Les objectifs de notre critique formelle seront les suivants :

\begin{itemize}
\item Analyser les différentes composantes du projet longue durée
\item Déceler les points forts et les points faibles du projet afin d'émettre une opinion sur sa continuité
\end{itemize}

Les axes d'analyse seront les suivants :

\begin{itemize}
\item L'idée du projet
\item Les prévisions d'activités
\item Les moyens techniques et humains mis en oeuvre
\item Synthèse de l'analyse critique
\end{itemize}

\section{L'idée du projet}

L'idée derrière la réalisation du projet d'ingénierie des sites isolés est assez vaste pour donner à une équipe l'occasion de réaliser un vrai travail en collaboration, s'étendant sur une durée assez conséquente, et elle est également assez précise pour limiter les risques de divergence par rapports aux objectifs du projet, et pour faciliter la tâche aux enseignants qui doivent veiller à apporter le soutien et les conseils nécessaires au bon déroulement des activités.

Travailler sur ce projet a permis à notre équipe d'en apprendre plus à la fois sur elle-même que sur les divers mondes techniques de l'ingénierie. D'un point de vue Qualité, le dimensionnement ambitieux a permis la mise en place d'une politique Qualité plus solide et plus généraliste qu'avant, à travers la mise en place de processus qualité plus clairement définis, et par l'investissement de temps et de ressources dans l'apprentissage de nouveaux outils plus performants tels que Git et LaTeX.

\section{Une roadmap respectée}

Dans leurs grandes lignes, les grandes étapes du projet ont été respectées à la lettre. Le cheminement dans l'ébauche et la réalisation des documents, GEI, Chef de Projet et Responsable Qualité en même temps, a été facilité par les informations prévisionnelles fournies dans le sujet de l'appel d'offre, et par les estimations concues lors de l'initialisation du projet. Cet effort d'acquérir une vision globale du projet et un survol temporel des différentes étapes amenant le projet de son initialisation jusqu'à sa terminaison par remise des livrables, est un effort critique qu'il est important de ne pas sous-estimer. Il permet de dimensionner et de chiffrer le projet avec précision, et il joue également un rôle de guide dans la réalisation des différents sous-projets qui le composent, lors de laquelle il est facile de perdre son recul et de dévier des objectifs principaux vers des points annexes, voir des détails qui pourraient apparaître comme insignifiants en retrospective du projet.

\section{Une qualité toujours grandissante}

Par son caractère ambitieux en terme de scope et d'envergure, ce projet a également fourni un cadre propice à l'accentuation des traits de qualité : Il a donné lieu au raffinement des principes de la Qualité dans un contexte de projet, et a permis une meilleure mise en place d'une Assurance Qualité plus solide en terme de spécification et de compréhension par les différents acteurs du projet. Permettant également quelques expérimentations au niveau Qualité, comme l'utilisation de nouveaux outils et des essais d'amélioration des processus de conception de notre équipe, il nous a donné l'occasion de parfaire notre façon de travailler et d'approcher encore plus près de notre objectif 100\% Qualité.

\section{Conclusion}

Ce projet présente des intérêts formateurs notables, adapté à un cadre d'apprentissage. Pour notre équipe, il a été le premier d'une série de projets à envergure plus importante qu'à l'accoutumée, et il nous a permis de nous préparer anticiper des projets de tailles variées. Il va de soi qu'on n'appréhende pas de la même façon deux projets aux dimensions différentes, et c'est une vision qui nous avons acquise avec succès pendant le développement de ce projet longue durée.

\end{document}
