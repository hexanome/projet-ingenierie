\documentclass[a4paper]{article}

\usepackage[utf8]{inputenc}   
\usepackage[top=2cm, bottom=2cm, left=2cm, right=2cm]{geometry}
\usepackage{ucs}
% Reconnaitre les caratères accentués dans le source.
\usepackage[T1]{fontenc} 
\usepackage{lmodern}
\usepackage[francais]{babel}
% Insertion d'images
\usepackage{graphicx}

\begin{document}

\section{Glossaire}

Ce glossaire est issu de la norme AFNOR Z 67-100-3 et adapté au contexte du projet de monitoring de systèmes isolés.

\subsection{A}

\begin{itemize}
\item AQ : Assurance Qualité
\end{itemize}

\subsection{C}

\begin{itemize}
\item CdC : Cahier des Charges
\item CdP : Chef de Projet
\item COPEVUE : Comité pour la protection de l’environnement de l’UE
\item CQ : Contrôle Qualité
\end{itemize}

\subsection{E}

\begin{itemize}
\item EMP : Electro Magnetic Pulse : impulsion électromagnétique ou brève émission d’ondes radio
de très forte amplitude capable de perturber ou d’endommager
tout équipement électronique
\end{itemize}

\subsection{G}

\begin{itemize}
\item GEI : Groupe d’Etudes Informatique
\item GPRS : General packet radio Service : norme de téléphonie mobile dérivée du GSM permettant
l’échange de données par paquets avec facturation au volume
\item GPS : Global Positioning System : système de positionnement par satellite développé par la
défense américaine
\item GSM : Global System for Mobile communications :
norme de transmission radio numérique de seconde génération
utilisée pour la téléphonie mobile
\end{itemize}

\subsection{I}

\begin{itemize}
\item IHM : Interface Homme/Machine
\end{itemize}

\subsection{M}

\begin{itemize}
\item MOA : Maîtrise d’Ouvrage
\item MOE : Maîtrise d’Oeuvre
\end{itemize}

\subsection{P}

\begin{itemize}
\item PAQL : Plan d’Assurance Qualité Logiciel
\item PAQP : Plan d’Assurance Qualité Projet
\item PMP : Plan de Management de Projet
\end{itemize}

\subsection{R}

\begin{itemize}
\item RQ : Responsable « Méthodes et Qualité »
\end{itemize}

\subsection{S}

\begin{itemize}
\item STB : Spécification technique des besoins
\item Système embarqué : Système informatique encapsulé dans un équipement plus large dans
lequel il assure une tâche spécifique (contrôle, surveillance…) sans
intervention humaine
\end{itemize}

\subsection{U}

\begin{itemize}
\item UE : Union Européenne
\end{itemize}

\end{document}
