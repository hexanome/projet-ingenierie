
\documentclass[a4paper]{article}

\usepackage[utf8]{inputenc}   
\usepackage[top=2cm, bottom=2cm, left=2cm, right=2cm]{geometry}
\usepackage{ucs}
% Reconnaitre les caratères accentués dans le source.
\usepackage[T1]{fontenc} 
\usepackage{lmodern}
\usepackage[francais]{babel}
% Insertion d'images
\usepackage{graphicx}

\begin{document}

\part{Introduction}

\section{Objectifs}

L'objectif de ce document est de faire une réflexion sur les ``bonnes pratiques'' dans le cadre de notre projet
d'ingénierie. Notre reflexion portera dans un premier temps sur la rédaction d'une procédure pour le Chef de Projet (CdP),
et dans un second temps sur la rédaction d'un Cahier des Charges (CdC) de sous-ensemble logiciel.

\part{Best Practise 1 - Rédaction d'une procédure}

\section{Procédure}

% Logigramme

La méthode se divise en deux grandes phases :

\begin{itemize}
\item Une phase d’analyse, où les exigences seront déterminées ;
\item Une phase de suivi, où toutes les actions entreprises seront assignées à une exigence, afin de tracer quelle fonctionnalité répond à quelle volonté du client.
\end{itemize}

La phase d’analyse consiste à récolter les exigences parmi toutes les sources disponibles : documents, entretiens avec le client, etc. Elle se déroule en quelques étapes décrites sur le schéma page suivante.

\paragraph{Détecter une exigence}
Ici, il s’agit de détecter une exigence pendant la lecture d’un document, la synthèse du rendez-vous client, etc. Les exigences peuvent être détectées en une fois puis formalisées ensuite, ou formalisée une à une.

\paragraph{Classer et hiérarchiser}
Ici, il s’agit de :

\begin{itemize}
\item Repérer si l’exigence est associée à une autre déjà classée (exigences mères et filles) ;
\item Attribuer un identifiant unique permettant de tracer l’exigence ;
\item Éventuellement, diviser l’exigence en éléments plus simple si celle-ci est complexe.
\end{itemize}

\paragraph{Formaliser l’exigence}
Ici, il s’agit de décrire plus profondément l’exigence. On lui donne d’abord un libellé, qui permettra au lecteur de comprendre rapidement de quoi il retourne, puis on écrit un petit paragraphe descriptif plus complet, qui synthétise l’exigence.


\paragraph{Estimer la priorité et difficulté}
Enfin, on donne une priorité de traitement à l’exigence, selon quelle soit jugée critique par le client ou non.  On estime également rapidement la difficulté d’implémentation de l’exigence en première approche, afin d’avoir une idée du travail à faire.

\section{Outils}

Nous proposons comme outil de travail une feuille de calcul telle que ci-dessous pour répertorier les exigences.

\part{Best Practise 2 - Rédaction d'un cahier des charges logiciel}

\end{document}
