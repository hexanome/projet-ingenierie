%-------------------------------------------
% En-tête type de document pour le projet PLD
% Il suffit de remplir le input ligne 45
%-------------------------------------------

\documentclass[a4paper]{article}

\usepackage[utf8]{inputenc}   
\usepackage[top=2cm, bottom=2cm, left=2cm, right=2cm]{geometry}
\usepackage{ucs}
% Reconnaitre les caratères accentués dans le source.
\usepackage[T1]{fontenc} 
\usepackage{lmodern}
\usepackage[francais]{babel}
% Insertion d'images
\usepackage{graphicx}
% Utilisation du symbole EURO
\usepackage{eurosym}

\begin{document}

%------------------------------------- Page de titre
\begin{titlepage}
~ 
\vfill
	\begin{center}
		\begin{Huge}
		Projet d'Ingénierie : Draft de PAQP\\
		\end{Huge} 
\vfill
		\textbf{Hexanome 4111 :} 
		\\Quentin \bsc{Calvez}, Matthieu \bsc{Coquet}, 
		\\Jan \bsc{Keromnes}, Alexandre \bsc{Lefoulon}, 
		\\Thaddée \bsc{Tyl}, Xavier \bsc{Sauvagnat},
		\\Tuuli \bsc{Tyrväinen}
\vfill		
		\begin{Large}
		Février 2012
		\end{Large}
\vfill
	\begin{tabular}{|c|c|c|c|c|}
 	 \hline
   Destinataire & Version & Etat & Dernière révision & Equipe \\
   \hline
   Client & 0.1 & Draft & \today & H4111 \\
   \hline
	\end{tabular}
	\end{center}
\vfill
\end{titlepage}
%----------------------------------------------------
%--------------------------------- Table des matières
\newpage
\tableofcontents
\newpage
%----------------------------------------------------

% TODO s'adresse à des non techniciens pour leur présenter la solution
% TODO schéma de la solution expliqué
% TODO annexe référence aux trois dossiers GEI

\section{Préliminaires}

\subsection{Cadre du Plan d'Assurance Qualité Projet (ou PAQ niveau Système)}

Le PAQP est mis en place dans le cadre de la réponse par l'hexanôme H4111 (MOE) à l'appel d'offre "Système de monitoring à distance de sites isolés" lancé par le COPEVUE.

\subsection{Logiciels concernés par le Plan d'Assurance Qualité Projet (PAQP)}

\subsection{Responsabilités associées au Plan d'Assurance Qualité Projet}
\subsection{Procédures d'évolution du PAQP}

Tout le monde peut être force de proposition pour faire évoluer le PAQP. Le PAQP est un document qui par nature est régulièrement amélioré. L'objectif de ce document est d'assurer la bonne qualité du projet, et d'approcher le ``Zero Defaut''.

Le PAQP peut-être amené à évoluer pour plusieurs raisons:

\begin{itemize}
\item détection d'un défaut, d'une imprécision ou d'une faille dans le PAQP
\item découverte d'une "Best-Practice" qui peut être source d'inspiration et de modèle pour le présent PAQP
\item réflexion et mise en place d'une nouvelle idée.
\end{itemize}

Toute procédure d'évolution du PAQP doit être soumise au RQ, qui la prendra en considération, et qui devra être validée par le CdP.

Lorsqu'une procédure d'évolution du PAQP aboutit, tous les membres du projet sont avertis et informés.

\subsection{Procédure à suivre en cas de non-application du  PAQP}

Lorsqu'un document, résultat ou livrable produit par l'équipe du projet ne respecte pas le PAQP, il ne pourra pas être validé. Ceci est une règle essentielle.

L'auteur de la non-conformité ou de l'écart par rapport à la référence sera averti par le RQ et/ou le CdP, et il lui sera fourni les éléments et informations nécessaires à la correction. Ce dernier devra alors prendre en compte ces informations, et procéder aux modifications nécessaires, pour que le document, résultat ou livrable produit puisse être définitivement validé.

\section{Introduction}

\section{Documents de Référence et applicables}

% best practise, procédure

\subsection{Documents de référence}

\subsection{Documents Applicables}

\section{Terminologies et abréviations}

\section{Organisation humaine du comité de pilotage du projet}

\subsection{Rôle des différents intervenants (Intervenants pour la maîtrise d'ouvrage, intervenants pour la maîtrise d'œuvre)}

\subsection{Relations entre les intervenants}

\subsection{Planning des réunions et règles}

\section{Qualité au niveau du Processus}

\subsection{Présentation de la démarche de développement au niveau Projet (ou Système)}

\subsubsection{Généralités}

Le développement de ce système se basera sur le "cycle en V", qui produit des livrables à la fin de chacune des phases du cycle.

Ceci permettra de pouvoir valider les livrables produits avant de passer à l'étape suivante en cas de validation, ou alors de recommencer jusqu'à validation dans le cas inverse.

% TODO Diagramme du "Cycle en V" du cours.

\subsubsection{Phase d'Etude Préalable}

\subsubsection{Phase d'Etude Détaillée}

\subsubsection{Phase d'Intégration Système}

\subsubsection{Phase de Validation Système}

\subsubsection{Phase de mise en œuvre sur site pilote}

\subsection{Règles de Qualité pour l'ingénierie concurrente}

\subsubsection{Règles sur la rédaction d'un Cahier des Charges d'un sous-projet}

\subsubsection{Règles sur la définition précise des résultats attendus pour chaque sous-projets}

\subsubsection{Règles sur le suivi qualité des sous-projets}

\subsubsection{Règles sur la définition de critères d'acceptation des sous-projets avant intégration}

\subsection{Présentation des démarche de développement au niveau sous-projets (niv. réalisation)}

\subsubsection{Liste des processus de développement susceptibles d'être retenus pour le développement des sous-projets}

\subsubsection{Description du cycle de développement numéro 1}

Liste des étapes
Pour étape numéro J
Documents en entrée
Documents en sortie
Conditions de validation de l'étape
      -     Suivi de projet

\section{Documentation (Règles communes au projet)}

\subsection{Structuration de la documentation}

\subsection{Liste des documents de gestion de projet}

\subsection{Liste des documents relatifs à la qualité}

\subsection{Liste des documents techniques et de réalisation}

\subsection{Manuels d'utilisation et de mise en œuvre}

\section{Gestion de configuration (Règles communes au projet)}

\subsection{Conventions d'Identification}

\subsection{Procédures d'identification des éléments de configuration }

\subsubsection{Responsabilités}

Le RQ sera responsable de la mis en place de l'outil de gestion de configuration, de ses réglages et de sa maintenance.

Les différents membres du projet devront maîtriser l'outil. Pour cela, se référer à la documentation officielle de l'outil.

\subsubsection{Procédures de gestion de la configuration (Gestion des Eléments de Configuration Logiciel et des Configurations de Référence)}

\subsection{Gestion des ressources partagées}

\paragraph{Remarques}

Mettre l'arbre de découpage en Articles de configuration logicielle du Projet en Annexe

\section{Gestions des modifications (Règles communes au projet)}

\subsection{Origines des modifications}

\subsection{Procédures et organisation des modifications}

\section{Méthodes, Outils et Règles (Règles communes au projet)}

\subsection{Méthodes}

\subsection{Outils (Outils logiciels, autres outils)}

\subsection{Règles et normes (Documentation, programmation)}

\section{Contrôle des Fournisseurs}

\subsection{Exigences vis-à-vis des sous-traitants}

\subsection{Exigences vis-à-vis des co-traitants}

\subsection{Logiciels achetés, loués ou imposés}

\section{Reproduction, Protection, Livraison (au niveau projet)}

\subsection{Précautions à prendre lors de la reproduction}

\subsection{Précautions prises pour assurer le stockage des logiciels}

\subsubsection{Protection des données contre les incidents}

\subsubsection{Protection des données contre les agressions extérieures}

\subsubsection{Autres précautions}

\subsection{Modalités de livraison}

\subsubsection{Délais}

\subsubsection{Installation}

\subsubsection{Formation}

\subsubsection{Migration de l'ancien vers le nouveau système}

\section{Suivi de l'application du Plan Qualité}

\subsection{Principes}

L'application du plan qualité est primordiale si l'on souhaite effectuer un travail de qualité et produire des livrables respectant une certaine homogénéité et cohérence.

L'assurance qualité concerne toutes les procédures qualité établies par le RQ.

\subsection{Interventions du Responsable Qualité (Niv. Projet) sur la démarche de Développement}

Lors des différentes phases de développement du projet, le RQ a pour principales responsabilités: - Le support qualité auprès de l'équipe projet - la validation de la forme des documents produits et livrés selon les règles énoncées dans la Gestion de la Documentation. - la vérification du suivi et de l'application du PAQP par l'équipe projet - la création, le maintien et l'évolution du Système Qualité.

\subsection{Modalités de réception des résultats des sous-projets avant intégration (description par sous-projets)}

\section{Conclusion}

Ce PAQP est un document et un outil qui permet de garantir une solution finale de qualité, à condition qu'il soit bien appliqué.

Il permet également d'assurer que les attentes du client (COPEVUE) vont être prises en compte.

La Qualité est toujours en évolution, et a pour vocation d'être toujours améliorée. C'est pourquoi le PAQP (le présent document) peut être sujet à modification.

\end{document}
