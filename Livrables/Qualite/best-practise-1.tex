
\documentclass[a4paper]{article}

\usepackage[utf8]{inputenc}   
\usepackage[top=2cm, bottom=2cm, left=2cm, right=2cm]{geometry}
\usepackage{ucs}
% Reconnaitre les caratères accentués dans le source.
\usepackage[T1]{fontenc} 
\usepackage{lmodern}
\usepackage[francais]{babel}
% Insertion d'images
\usepackage{graphicx}

\begin{document}

\title{Bonne Pratique 1 : Rédaction d'une Procédure}
\maketitle

\section{Rappel}

Une procédure est un \textbf{document} décrivant une \textbf{activité spécifique} du système concerné en précisant les \textbf{responsabilités}, les \textbf{interactions} entre les services et les \textbf{moyens} requis pour obtenir le résultat prévu.

\textbf{L'intérêt} de l'existence de procédures vis-à-vis des clients et des organismes de certification est \textbf{double} puisqu'il permet de comparer des façons de faire procéder par rapport aux \textbf{autres entreprises} et par rapport à \textbf{des normes}. La procédure ainsi créée permettra de fournir un \textbf{référentiel commun} non seulement entre la MOE et la MOA mais aussi entre les membres du groupe de travail afin d'obtenir un cadre de travail normalisé.

\section{Objet et domaine d'application}

% TODO schema QQOQCP

La meilleure méthode pour trouver facilement l'objet et le domaine d'application est de répondre aux questions suivantes; \textbf{"QQOQCP"}, c'est à dire :

\begin{itemize}
\item \textbf{Quoi?} de quoi parle cette procédure?
\item \textbf{Qui?} pour qui elle est décrite?
\item \textbf{Où, Quand?} dans quel cadre l'utiliser, dans quel document?
\item \textbf{Comment?} comment utiliser cette procédure?
\item \textbf{Pourquoi?} pourquoi utiliser cette procédure?
\end{itemize}

\section{Description de la procédure}

Le coeur de la description d'une procédure est \textbf{le logigramme}. Un logigramme permet d'être \textbf{clair et concis}. Il est cependant très important de respecter un certain formalisme. Pour aider à la compréhension d'un point sensible ou pour approfondir un point en particulier, il est peut être utile de joindre à ce logigramme des petits textes explicatifs.

% TODO Exemples de logigrammes

\section{Proposition de plan type pour une procédure}

\begin{itemize}
\item Objectifs, Domaine d'application et Références

\begin{itemize}
\item Objet: rappelle les raisons du pourquoi de la procédure (généralement les raisons concernant des exigences requises et les méthodes à mettre en oeuvre pour maîtriser un processus)
\item Domaine d'application: cette rubrique indique qui, dans l'organisation, est concerné par cette procédure
\item Références: cette rubrique fait référence à des documents internes ou externes
\end{itemize}

\item Terminologies et Abréviations

Définition des termes techniques les plus importants et étant capitaux pour la bonne compréhension de la procédure. Il est à noter que ces termes devront également se retrouver dans le Glossaire général du projet.

\item Le logigramme
\item Le texte d'accompagnement du logigramme
\item Les annexes documentaires, lorsque nécessaire
\end{itemize}

% TODO liens / annexes vers des exemples de la procédure

\end{document}
