
\documentclass[a4paper]{article}

\usepackage[utf8]{inputenc}   
\usepackage[top=2cm, bottom=2cm, left=2cm, right=2cm]{geometry}
\usepackage{ucs}
% Reconnaitre les caratères accentués dans le source.
\usepackage[T1]{fontenc} 
\usepackage{lmodern}
\usepackage[francais]{babel}
% Insertion d'images
\usepackage{graphicx}

\begin{document}

\part{Introduction}

\section{Cadre de l’étude et objectif}

\paragraph{Objectif}
Le dossier de synthèse a pour objectif de promouvoir le résultat de l’analyse et de la
conception du futur système (Dossier n° 3 dans le cas du « projet d’ingeniérie »). Il faut savoir
à qui s’adresse ce dossier de synthèse (ex. : comité de pilotage) et à quel usage est-il destiné
(ex . : prendre une décision pour poursuivre ou non le projet => par ex. dans le cas du « projet
d’ingeniérie », c’est pour être ou non retenu suite à l’appel d’offre puisque vous faites une réponse).

\section{Rappel synthétique des critiques de l’existant}



\part{Présentation de la solution proposée}

\section{Architecture}

\section{Fonctionnement général}

\section{Apports de la solution proposée}

\subsection{Thèmes de progrès}

\subsection{Coûts de la solution}

\part{Mise en place du système}

\section{Description de la mise en place du système}

\section{Plan de formation}

\section{ROI}

\part{Annexes}

\section{Organisation générale du système}

\end{document}
