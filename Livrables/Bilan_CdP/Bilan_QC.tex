Ce projet, avec le projet de système d'information (SPIE), a été un des premiers à nous mettre réellement en situation de bureau d'étude devant proposer une solution complête et mesurée à un problème concret. Nous avons donc étés confrontés à des problématiques qui nous étaient inhabituelles, et qui sont spécifiques au développement d'un système dans son ensemble. Le fait de devoir prendre en compte non seulement l'aspect matériel et logiciel d'un système, mais aussi ses conditions de fonctionnement, son alimentation, sa maintenance, et les coûts associés à chacun de ces aspects sont, en effet, autant de nouvelles dimensions qui viennent s'ajouter à la compléxité de la réponse au problème initial.

D'un point de vue plus personnel, j'ai trouvé ce projet intéressant de par sa complémentarité avec le projet Ghome, qui lui nous fait voir ``l'autre côté du mirroir'', à savoir le développement à proprement parler d'un système intéragissant avec des capteurs, et prenant des décisions en conséquence. Ce projet-ci au contraire nous permet de mieux apréhender toutes les étapes qui prennent place autour de la phase de développement, afin de s'assurer du développement d'une solution répondant au mieux à des besoins précis. Il nous permet de mieux nous rendre compte de ce que peut représenter le déploiement d'une solution industrielle réelle, et son application directe et réaliste aide à mieux s'immerger dans le rôle de consultant qui était le notre chaque mercredi matin. Au final, la principale frustration sur ce genre de projets restera celle de ne pas pouvoir aller jusqu'au bout et voir fonctionner le système une fois celui-ci réalisé.

En regard de la charge de travail, je n'ai pas trouvé celle-ci excessive, mais la réalisation correcte de l'ensemble des livrables demandés implique tout de même un important investissement en dehors des séances de travail. Les difficultés rencontrées lors de la rédaction des documents étaient souvent plus liées à déterminer précisement ce qui était attendu de nous, et n'étaient que rarement le résultat d'un manque de données sur lesquelles nous fonder. Nous avons ainsi perdu un temps précieux à se mettre d'accord sur la substance à apporter dans certains paragraphe, chacun interprétant la consigne à sa manière.

En terme de suggestions d'améliorations, une idée que j'aurai trouvé intéressante aurait été de plus jouer sur l'aspect ``concurrence sur un appel d'offre'' des différentes équipes, en les montant plus les unes contre les autres. L'aspect de compétition aurait, je pense, pu aporter une motivation supplémentaire qui aurait poussé les différents membres de l'équipe à essayer de chercher plus loins pour trouver des solutions innovantes et adaptées au problèmes qui nous étaient posé.