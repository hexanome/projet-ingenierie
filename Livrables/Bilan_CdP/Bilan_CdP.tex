\section{Bilan du chef de projet}
\subsection{Bilan horaire}
\subsection{Impressions personnelles}
Ce projet est pour notre hexanôme assez nouveau, il s'inscrit cependant dans l'évolution de notre cursus d'ingénieur informatique. Nous prenons conscience que les spécifications techniques des produits sont indispensables lorsque l'on veut déployer une solution à grande échelle. Alors que l'informatique s'affranchit souvent des contraintes de distance, de météo ou d'environnement nous nous sommes replacés ici dans un contexte plus industriel qui imposait que notre solution respecte de fortes contraintes d'autonomie énergétique et logicielle. Bien que le domaine technologique soit différent, ce projet était assez lié avec le PLD SI, ce qui pouvait parfois décourager les membres du GEI devant la quantité assez conséquente de documents à rédiger. L'intérêt de ce projet aura été sans nul doute de poser sur le papier les moyens à mettre en place pour réaliser un système de monitoring complet. Nous prenons en considération le fait que chaque brique matérielle ou logicielle doit être parfaitement spécifiée si l'on veut palier aux problèmes mais aussi faciliter la réalisation ou le déploiement du projet. C'était pour ainsi dire la première fois qu'un projet si important nous était confié et la partie organisationnelle est conséquente à gérer pour le chef de projet notamment dans la synchronisation des équipes de travail et dans la motivation qu'il faut insufler au fur et à mesure de l'avancement. Par exemple, en termes de synchronisation, il était nécessaire que certains documents soient réalisés en priorité par le responsable qualité afin que ceux-ci puissent servir pour la rédaction de certains documents chez les GEI. Pour ma part, j'aurai attaché beaucoup d'importance à la communication au sein de l'hexanôme de travail car il est nécessaire que chacun dispose de la même information afin de construire des bases de travail communes. J'essayais de m'adresser le plus souvent au groupe entier afin que chacun prenne conscience des impératifs du reste du groupe de travail. Ainsi, lors d'éventuels conflits, les choses sont claires et posées et le GEI peut comprendre pourquoi le RQ est en retard par exemple car il connait les enjeux d'une bonne gestion de la qualité pour la production de livrables.

D'un point de vue charge de travail, je considère que celle-ci est tout de même assez importante (voir trop) notamment car les livrables à remettre sont assez proches des livrables à remettre pour le pld SI. L'équipe GEI me l'a fait savoir à de nombreuses reprises et mon but était donc de créer une émulation nouvelle à chaque début de séance de travail autour de ce projet en emmenant le GEI vers des sujets qu'il maitrisait et qu'il appréciait. Je pense que notre hexanôme dispose de très grandes qualités de par sa compréhension fine et sa connaissance des enjeux technologiques mis en oeuvre dans ce projet (autonomie, performance, portabilité). Nous avons cependant parfois eu du mal à poser nos idées sur le papier et à expliquer avec le plus de détails possible la manière dont nous allions mettre en oeuvre ce système. Je me rends compte de l'importance de l'esprit de synthèse dans le cadre de la production de livrables destinés à un client qui parfois ne maitrise que peu de concepts informatiques. Le fait de pouvoir expliquer clairement et de manière complète la réponse à un problème est donc une qualité que l'ingénieur doit mettre en place afin de valoriser son travail et de pouvoir convaincre un client que son travail est le meilleur.

Je considère cette expérience en tant que chef de projet comme très enrichissante notamment car on se rend compte des éventuelles erreurs plus que des réussites et ceci permet de tenir une "black list" des pratiques à ne plus reproduire que ce soit dans le discours ou dans la manière de gérer l'hexanôme.
Je remercie les professeurs/clients d'avoir porté un oeil critique sur les intervention du CdP en commentant certaines action en temps réel. Ceci m'a permis de redresser le tir lors de critiques \og négatives \fg mais aussi et surtout de prendre confiance en moi lors de critiques \og positives \fg. La communication directe permet d'avoir un feedback que certains professeurs se refusent parfois de donner.



\section{Bilan du RQ, Jan Keromnes}
\section{Bilan des GEI}
\subsection{Quentin Calvez}
Ce projet, avec le projet de système d'information (SPIE), a été un des premier à nous mettre réellement en situation de bureau d'étude devant proposer une solution complète et mesurée à un problème concret. Nous avons donc été confrontés à des problématiques qui nous étaient inhabituelles, et qui sont spécifiques au développement d'un système dans son ensemble. Le fait de devoir prendre en compte non seulement l'aspect matériel et logiciel d'un système, mais aussi ses conditions de fonctionnement, son alimentation, sa maintenance, et les coûts associés à chacun de ces aspects sont, en effet, autant de nouvelles dimensions qui viennent s'ajouter à la complexité de la réponse au problème initial.

D'un point de vue plus personnel, j'ai trouvé ce projet intéressant de par sa complémentarité avec le projet Ghome, qui lui nous fait voir `\og l'autre côté du mirroir \fg, à savoir le développement à proprement parler d'un système intéragissant avec des capteurs, et prenant des décisions en conséquence. Ce projet-ci au contraire nous permet de mieux apréhender toutes les étapes qui prennent place autour de la phase de développement, afin de s'assurer du développement d'une solution répondant au mieux à des besoins précis. Il nous permet de mieux nous rendre compte de ce que peut représenter le déploiement d'une solution industrielle réelle, et son application directe et réaliste aide à mieux s'immerger dans le rôle de consultant qui était le notre chaque mercredi matin. Au final, la principale frustration sur ce genre de projets restera celle de ne pas pouvoir aller jusqu'au bout et voir fonctionner le système une fois celui-ci réalisé.

En regard de la charge de travail, je n'ai pas trouvé celle-ci excessive, mais la réalisation correcte de l'ensemble des livrables demandés implique tout de même un important investissement en dehors des séances de travail. Les difficultés rencontrées lors de la rédaction des documents étaient souvent plus liées à déterminer précisement ce qui était attendu de nous, et n'étaient que rarement le résultat d'un manque de données sur lesquelles nous fonder. Nous avons ainsi perdu un temps précieux à se mettre d'accord sur la substance à apporter dans certains paragraphe, chacun interprétant la consigne à sa manière.

En terme de suggestions d'améliorations, une idée que j'aurai trouvé intéressante aurait été de plus jouer sur l'aspect \og concurrence sur un appel d'offre \fg des différentes équipes, en les montant plus les unes contre les autres. L'aspect de compétition aurait, je pense, pu aporter une motivation supplémentaire qui aurait poussé les différents membres de l'équipe à essayer de chercher plus loins pour trouver des solutions innovantes et adaptées aux problèmes qui nous étaient posés.

\subsection{Matthieu Coquet}
Le projet d’ingénierie est un projet important de la 4IF, et est surement un des plus représentatifs du monde de l’entreprise. Le résultat est un vrai plus dans notre formation, où la rédaction de documents similaire nous sera régulièrement demandé. 
Ce projet a été intéressant car il nous permet de répondre à une problématique intéressante et au gout du jour. Pouvoir jeter un œoeil aux projets de nos prédécesseurs nous a permis de ne pas rester bloqué et de voir plusieurs approches pour rédiger un document. 
Rédiger les documents a quelques fois été difficile, il est par exemple délicat de réaliser un cahier des charges précis et détaillé pour un projet dont certains points restes flous. C’est un exercice laborieux car nous n’avions jamais réalisé des documents similaires dans notre formation. On observe également des similarités avec le projet SPIE.

\subsection{Thaddee Tyl}
Il va de soi que ce projet s'est déroulé sans déboires majeurs.  En effet, nous
n'avons manqué ni d'encadrement, ni d'idées.  Toutefois, il est regrettable que
nous n'ayons pas eu accès à un dossier en réponse à un appel d'offre, venant
réellement du monde de l'entreprise, en vue de jauger notre travail à l'aune
d'un produit professionnel.  De surcroît, l'exigence d'une liste de livrables
bien définie, non content de n'être pas réaliste (quel commanditaire donne une
telle précision dans la forme que doit avoir l'offre?), gêne à la conception
d'un dossier vraiment original, vraiment frappant.  Nous avons des idées de mise
en forme et de présentation, mais ce n'est que par le truchement d'une pile de
livrables prédéfinis que nous pouvons nous exprimer.

Si je pouvais recommander des améliorations, elles seraient les suivantes:
davantage de liberté dans la forme, davantage de professionnalisme sur le fond.
Cela garantirait au moins que notre rôle ne se réduise pas à aligner paragraphe
sur paragraphe afin de remplir le compte du nombre de pages requis par livrable.

