Ce projet est pour notre hexanôme assez nouveau, il s'inscrit cependant dans l'évolution de notre cursus d'ingénieur informatique. Nous prenons conscience que les spécifications techniques des produits sont indispensables lorsque l'on veut déployer une solution à grande échelle. Alors que l'informatique s'affranchit souvent des contraintes de distance, de météo ou d'environnement nous nous sommes replacés ici dans un contexte plus industriel qui imposait que notre solution respecte de fortes contraintes d'autonomie énergétique et logicielle. Bien que le domaine technologique soit différent, ce projet était assez lié avec le PLD SI, ce qui pouvait parfois décourager les membres du GEI devant la quantité assez conséquente de documents à rédiger. L'intérêt de ce projet aura été sans nul doute de poser sur le papier les moyens à mettre en place pour réaliser un système de monitoring complet. Nous prenons en considération le fait que chaque brique matérielle ou logicielle doit être parfaitement spécifiée si l'on veut palier aux problèmes mais aussi faciliter la réalisation ou le déploiement du projet. C'était pour ainsi dire la première fois qu'un projet si important nous était confié et la partie organisationnelle est conséquente à gérer pour le chef de projet, notamment dans la synchronisation des équipes de travail et dans la motivation qu'il faut insufler au fur et à mesure de l'avancement. Par exemple, en termes de synchronisation, il était nécessaire que certains documents soient réalisés en priorité par le responsable qualité afin que ceux-ci puissent servir pour la rédaction de certains documents chez les GEI. Pour ma part, j'aurai attaché beaucoup d'importance à la communication au sein de l'hexanôme de travail car il est nécessaire que chacun dispose de la même information afin de construire des bases de travail communes. J'essayais de m'adresser le plus souvent au groupe entier afin que chacun prenne conscience des impératifs du reste du groupe de travail. Ainsi, lors d'éventuels conflits, les choses sont claires et posées et le GEI peut comprendre pourquoi le RQ est en retard par exemple car il connait les enjeux d'une bonne gestion de la qualité pour la production de livrables.

D'un point de vue charge de travail, je considère que celle-ci est tout de même assez importante (voir trop) notamment car les livrables à remettre sont assez proches des livrables à remettre pour le pld SI. L'équipe GEI me l'a fait savoir à de nombreuses reprises et mon but était donc de créer une émulation nouvelle à chaque début de séance de travail autour de ce projet en emmenant le GEI vers des sujets qu'il maitrisait et qu'il appréciait. Je pense que notre hexanôme dispose de très grandes qualités de par sa compréhension fine et sa connaissance des enjeux technologiques mis en oeuvre dans ce projet (autonomie, performance, portabilité). Nous avons cependant parfois eu du mal à poser nos idées sur le papier et à expliquer avec le plus de détails possible la manière dont nous allions mettre en oeuvre ce système. Je me rends compte de l'importance de l'esprit de synthèse dans le cadre de la production de livrables destinés à un client qui parfois ne maitrise que peu de concept informatiques. Le fait de pouvoir expliquer clairement et de manière complète la réponse à un problème est donc une qualité que l'ingénieur doit mettre en place afin de valoriser son travail et de pouvoir convaincre un client que son travail est le meilleur.

Je considère cette expérience en tant que chef de projet comme très enrichissante notamment car on se rend compte des éventuelles erreurs plus que des réussites et ceci permet de tenir une "black list" des pratiques à ne plus reproduire que ce soit dans le discours ou dans la manière de gérer l'hexanôme.
Je remercie les professeurs/clients d'avoir porté un oeil critique sur les intervention du CdP en commentant certaines action en temps réel. Ceci m'a permis de redresser le tirs lors de critiques \og négatives \fg mais aussi et surtout de prendre confiance en moi lors de critiques ``positives''. La communication directe permet d'avoir un feedback que certains professeurs se refusent parfois de donner.