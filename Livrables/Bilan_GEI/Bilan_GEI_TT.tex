Il va de soi que ce projet s'est déroulé sans déboires majeurs.  En effet, nous
n'avons manqué ni d'encadrement, ni d'idées.  Toutefois, il est regrettable que
nous n'ayons pas eu accès à un dossier en réponse à un appel d'offre, venant
réellement du monde de l'entreprise, en vue de jauger notre travail à l'aune
d'un produit professionnel.  De surcroît, l'exigence d'une liste de livrables
bien définie, non content de n'être pas réaliste (quel commanditaire donne une
telle précision dans la forme que doit avoir l'offre?), gêne à la conception
d'un dossier vraiment original, vraiment frappant.  Nous avons des idées de mise
en forme et de présentation, mais ce n'est que par le truchement d'une pile de
livrables prédéfinis que nous pouvons nous exprimer.

Si je pouvais recommander des améliorations, elles seraient les suivantes:
davantage de liberté dans la forme, davantage de professionnalisme sur le fond.
Cela garantirait au moins que notre rôle ne se réduise pas à aligner paragraphe
sur paragraphe afin de remplir le compte du nombre de pages requis par livrable.

Enfin, vous nous demandez de construire un cahier des charges.  Il me semble que
le cahier des charges est nécessaire à la préparation d'un appel d'offre ; ce
partant, le cahier des charges est déjà censé exister.
