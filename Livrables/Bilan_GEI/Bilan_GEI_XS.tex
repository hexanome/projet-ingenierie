Personnellemeent je me souviendrais de ce projet comme étant un moment important de la 4IF, de part sa complétude et sa longueur. En effet, en tant que GEI, de nombreuses aptitudes nous ont été demandés.  
Tout d'abord , le cadre de l'appel d'offres, qui pourrait très bien être réelle, donne une toute  autre dimension au projet, l'objectif étant bien d'apporter une réponse ,et la meilleure, à des clients. De plus, le découpage en  phase permettait une approche réfléchie sur le sujet, en commançant par une étude de faisabilité, afin d'appréhender les difficultés, puis en continuant sur les spécifications, pour définir clairement les besoins pour finalement développer un dossier de conception, synthétisant notre réflexion sur les solutions à apporter pour répondre aux problèmes posés.
Ceci m'a permis d'effectuer un vrai travail d'ingénieur, qui, face à un nouveau problème, certes dans notre domaine de compétence (contrairement à la découpe du cuir), grâce à une méthodologie et un raisonnement scientifique en équipe, peut apporter une réponse.
Tout ceci s'accompagne aussi de quelques désagréments, il faut bien le dire, la rédaction des toutes les compte-rendus étant un peu fastidieuse et représentant une lourde charge de travail et je trouve dommage que nous n'ayons pas un accès direct à des informations plus précises sur les composants industriels, car il faut bien souvent faire des demandes de devis pour obtenir les spécifications et prix des pièces (capteurs en autres...).