\section{Introduction}
\subsection{Présentation du projet}
Le COPEVUE a lancé un appel d'offre dans le cadre de la réalisation d'un système de monitoring de sites isolés. Il s'agit donc de concevoir en premier lieu une solution technique permettant de répondre au mieux aux exigences fonctionnelles et non fonctionnelles que le COPEVUE formule. De façon synthétique notre équipe va proposer une solution permettant de surveiller des sites naturels difficiles d'accès (souvent à cause des conditions environnementales) et peu peuplés. Dans ces sites isolés sont souvent regroupés des postes de travail et ces zones doivent pouvoir être surveillées en dépit de la distance qui les sépare du bureau de contrôle.

\subsection{Présentation du document}
Ce document permet de définir précisemment l'architecture applicative et technique de l'ensemble de notre solution, en rapport avec le dossier de Spécification Technique des Besoins. L'objectifs est d'apporté une réponse claire aux questions suivantes:
%Copier coller du doc de Regis
\begin{itemize}
	\item Quels sont les objets manipulés?
	\item Quelles sont les données manipulées?
	\item Analyse transformationnelle de ces données
	\item Description des stations locales et du système central
	\item Dimensionnement
	\item Analyse de la complexité
\end{itemize}

\section{Organisation générale du système}
%Un beau schéma de l'architecture générale

\subsection{Système central}


\section{Régles de pilotage du système}

\section{Architecture applicative}

\section{Architecture informatique et matérielle}

\section{Réflexions sur les données }

\subsection{Modèles de données du système}
, modèles de la base dedonnées intgrant les données de la prduction, volumétrie (évaluation de la base de données
pour une entreprise type en s’appuyant sur réflexion de l’annexe 1),
\subsection{Volume de données sur le réseau}

\section{Gestion des problèmes et anomalies, sécurité}

\section{Conclusion}

\section{Annexe 1 : Répresentation informatique des objets (messages…..)}

\section{Annexe 2 : Réflexion sur le réseau (principes qualité, pour la conception du réseau,
description du réseau (logiciel, couches…)…}

\section{Annexe 3 : Démarrage du système (petite réflexion de ¼ à ½ pages max.)}