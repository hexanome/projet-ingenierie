% instanciation de la gestion de la documentation
% droit de tenter des expériences pour l'organisation de la production (auto-critique)
% Annexes

\begin{document}

\part{Introduction}

\section{Rappel}

\paragraph{Documentation}
Documentation: Ensemble de documents relatifs à un projet - notice - mode d’emploi - action de sélectionner, classer, utiliser ou diffuser des documents.

La documentation d’un projet a une importance primordiale : c’est l’outil de communication et de dialogue entre les membres de l’équipe projet et les intervenants extérieurs (membre des comités de pilotage et utilisateurs, chef de projet, coordination des projets, utilisateurs, etc...). Elle assure aussi la pérennité des informations au sein du projet.

Afin d’organiser la gestion de la documentation produite par projet, il convient au préalable d’identifier tous les types de documents relatifs aux diverses étapes d’un projet, de les référencer de manière homogène pour ensuite définir un mode de gestion commun à tous les projets.

\section{Object du document}

Le dossier de gestion et organisation de la documentation du projet a pour objectif de définir l’ensemble des règles
communes concernant la gestion de la documentation (structuration, page de garde, cycle de vie d’un
document, gestion de version, structuration du système documentaire, sauvegardes et diffusion, ...) et
l’organisation de la documentation en définissant des plans types pour les documents relatifs à la gestion de
projet, les documents relatifs à la qualité et les annexes relatives aux études.

Nous proposerons des règles de gestion de la documentation pour ce projet qui permettront de mettre en oeuvre des moyens
de référenciation homogène de l'ensemble de la documentation relative au projet, d'en organiser la production, le classement
et l'accès.

\part{Gestion de la documentation}

Ce chapitre précise les règles de gestion de la documentation à mettre en oeuvre dans tout projet.

Pour mieux comprendre la nécessité d’une gestion rigoureuse de la documentation, il convient en premier lieu de détailler les états par lesquels passe un document avant d’être diffusé ainsi que le rôle des différents acteurs.

\section{Les acteurs et leurs responsabilités}

Les différents acteurs sont :

\begin{itemize}
\item le chargé de la gestion documentaire (généralement le responsable qualité du projet),
\item le(s) auteur(s) du document,
\item les responsables de la vérification (membres de l’équipe projet ou intervenants extérieurs),
\item les responsables de la validation (une ou plusieurs personnes désignées).
\end{itemize}

% tableau

\section{Cycle de vie d'un document}

Un document passe ou peut passer par un certain nombre d'états :

\begin{itemize}
\item travail : le document est en cours d'élaboration par l'auteur
\item terminé : le document satisfait l'auteur; il est prêt à être diffusé
\item vérifié (optionnel) : le document est approuvé par d'autres membres de l'équipe, des intervenants extérieurs et/ou le contrôle qualité
\item validé : le document est approuvé par les personnes habilitées et prend valeur de référence au sein du projet
\item périmé : le document n’est plus adapté et est donc retiré à tous ses détenteurs (retrait d'usage)
\item archivage : le document n'est plus consulté régulièrement, mais une trace de son existence demeure (pour une durée définie par le chargé de gestion de la documentation du projet)
\item destruction : le document n'est pas archivé ou le délai d'archivage est écoulé
\end{itemize}

% schema

\subsection{Production du document}

Un document en cours de production est dans l'état ``travail''.

Lorsque l'auteur obtient une rédaction qui le satisfait et ne souhaite plus apporter de modifications, il l'indique en le faisant passer à l'état ``terminé''.
Avant de faire passer un document en l'état ``terminé'', l'auteur peut le soumettre à des lectures croisées au sein de son équipe.

\subsection{Vérification / validation du document}

L'auteur diffuse alors le document aux vérificateurs puis aux validateurs, ou directement aux validateurs (la vérification est optionnelle selon le type de document). La diffusion se fait sous format papier ou électronique (choisir le plus pratique).
Il joint à son document une fiche de relecture où les remarques éventuelles des vérificateurs ou validateurs sont formalisées (modifications souhaitées).
Toutes les remarques de fond sur le contenu du document (imprécision, ambiguïtés, incohérences...) doivent être consignées dans cette fiche sauf les remarques relatives à la forme du document (fautes de frappe, d'orthographe, problèmes de mise en page...) qui peuvent être signalées directement sur la copie papier du document.
 Si les modifications du texte sont importantes, elles sont juste référencées dans la fiche de relecture puis décrites directement sur une copie papier du document.


La fiche de relecture comporte les éléments suivants : 

\paragraph{Une partie renseignée par l'auteur (avant transmission au vérificateur/validateur)}

\begin{itemize}
\item nom du demandeur
\item date de la demande
\item nom et référence du document
\item date de retour pour les remarques
\item aspects à examiner (contenu, forme, totalité, partie...)

\paragraph{Une partie renseignée par le vérificateur/validateur}

\begin{itemize}
\item nom du vérificateur ou validateur
\item date de vérification ou validation
\item conclusion de la vérification ou validation :
\item document validé,
\item document validé après intégration des modifications par l'auteur,
\item document à revalider (nécessite un nouveau passage en vérification/validation après intégration des modifications par l'auteur),
\item liste des points à modifier dans le document (numéro de §, page, description de la modification ou référence à une annotation dans la copie papier du document jointe).
\end{itemize}

Cette fiche (ainsi qu'éventuellement le document annoté joint) est transmise à l'auteur.
L'auteur répond aux remarques émises par les relecteurs dans la colonne ``justification réponses'' de la fiche prévue à cet effet.
L'auteur conserve une copie papier de la fiche.

Si la vérification/validation est acceptée, le document passe à l'état ``vérifié''/``validé'', sinon il revient en état de ``travail''.
L'auteur du document est chargé d'indiquer en page de garde du document l'état dans lequel le document se trouve, ainsi que les noms des vérificateurs/validateurs et les dates de vérification/validation.

\paragraph{NB}

Pour chaque document à valider, une date de retour des remarques est convenue. Si aucun retour n'est parvenu à l'auteur à la date prévue, le document est considéré comme validé.

\subsection{Archivage du document}

\section{Identification et structure de la documentation}

\subsection{Identification}
\subsection{Structure}

\section{Gestion des versions - révisions}

\section{Outils de production de la documentation}

\section{Classement}

\section{Gestion physique des fichiers contenant les documents}
\subsection{Répertoires}
\subsection{Noms des fichiers}
\subsection{Procédures de sauvegarde et archivage}

\part{Organisation de la production}

\section{Documents de gestion de projet}
\section{Documents d’étude et développement}
\section{Documents relatifs à la mise en oeuvre}
\section{Documents relatifs à la qualité}

\part{Annexes}

% plans types

\end{document}
