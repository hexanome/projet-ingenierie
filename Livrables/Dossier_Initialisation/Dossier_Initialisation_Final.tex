%-------------------------------------------
% En-tête type de document pour le projet PLD
% Il suffit de remplir le input ligne 45
%-------------------------------------------

\documentclass[a4paper]{article}

\usepackage[utf8]{inputenc}   
\usepackage[top=2cm, bottom=2cm, left=2cm, right=2cm]{geometry}
\usepackage{ucs}
% Reconnaitre les caratères accentués dans le source.
\usepackage[T1]{fontenc} 
\usepackage{lmodern}
\usepackage[francais]{babel}
% Insertion d'images
\usepackage{graphicx}
% Utilisation du symbole EURO
\usepackage{eurosym}

\begin{document}

%------------------------------------- Page de titre
\begin{titlepage}
~ 
\vfill
	\begin{center}
		\begin{Huge}
		Projet D'ingéniérie : Dossier d'Initialisation\\
		\end{Huge} 
\vfill
		\textbf{Hexanome 4111 :} 
		\\Quentin \bsc{Calvez}, Matthieu \bsc{Coquet}, 
		\\Jan \bsc{Keromnes}, Alexandre \bsc{Lefoulon}, 
		\\Thaddée \bsc{Tyl}, Xavier \bsc{Sauvagnat},
		\\Tuuli \bsc{Tyrväinen}
\vfill		
		\begin{Large}
		Janvier 2012
		\end{Large}
\vfill
	\begin{tabular}{|c|c|c|c|}
 	 \hline
	\end{tabular}
	\end{center}
\vfill
\end{titlepage}
%----------------------------------------------------
%--------------------------------- Table des matières
\newpage
\tableofcontents
\newpage
%----------------------------------------------------

%------------------- Insertion du contenu du document

%Corps du document :
\begin{document}
    \maketitle
    \tableofcontents
    

    \chapter{Introduction}
    Voici un peu de texte.
    
    \chapter{Contexte du document}
    Voici un peu de texte.
    
    \chapter{Documents de référence}
    Voici un peu de texte.
    
    \chapter{Rappel du problème}
    Voici un peu de texte.
    \section{Le contexte}
    Voici un peu de texte de section.
    \section{Les objectifs}
    Voici un peu de texte de section.
    
    \chapter{Les contraintes générales}
    Voici un peu de texte.
    \section{Confidentialité}
    Voici un peu de texte de section.
    \section{Etude de l'existant}
    Voici un peu de texte de section.
    \section{Exigences non fonctionnelles}
    Voici un peu de texte de section.
    
    \chapter{Organisation du travail}
    Voici un peu de texte.
    \section{Chef de projet (CdP)}
    Voici un peu de texte de section.
    \section{Responsable Qualité, Méthode et Documentation}
    Voici un peu de texte de section.
    \section{Groupe d'études informatique}
    Voici un peu de texte de section.
    
    \chapter{Listes des livrables attendus}
    Voici un peu de texte.
    \section{Chef de projet}
    Voici un peu de texte de section.
    \section{Responsable Qualité}
    Voici un peu de texte de section.
    \section{Groupe d'études informatique}
    Voici un peu de texte de section.
    
    \chapter{Organigramme des tâches}
    Voici un peu de texte.
    \section{Macro-Phasage}
    Voici un peu de texte de section.
    \section{Diagramme de Gantt}
    Voici un peu de texte de section.
    
    \chapter{Modalités de suivi}
    Voici un peu de texte.
    \section{Les règles de suivi}
    Voici un peu de texte de section.
    \section{Les outils utilisés}
    Voici un peu de texte de section.
    \section{Procédures de révisions du planning}
    Voici un peu de texte de section.
    
    \chapter{Gestion des risques}
    Voici un peu de texte.
    \section{Risques concernant l'application du projet}
    Voici un peu de texte de section.
    \section{Risques propres au projet}
    Voici un peu de texte de section.
    
    \chapter{Conclusion}
    Voici un peu de texte.
    
    \chapter{Annexes}
    Voici un peu de texte.
    \section{Les fiches de tâche}
    Voici un peu de texte de section.
    \section{Exemple de fiche de revue}
    Voici un peu de texte de section.
    \section{Exemple de fiche de séance}
    Voici un peu de texte de section.



\end{document}
%----------------------------------------------------

\end{document}