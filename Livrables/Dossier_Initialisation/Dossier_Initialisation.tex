
%Corps du document :
\setlength{\parindent}{1cm}
\begin{document}
    \maketitle
    \tableofcontents
    

\chapter{Introduction}
Le but de ce dossier d'initialisation est de poser les bases nécessaires au bon déroulement du projet d'ingénierie. Nous répondons dans l'ensemble des documents produits par l'hexanôme H4111 à l'appel d'offre lancé par le COPEVUE visant à concevoir un Système de monitoring à distance de sites isolés. Dans ce document sont détaillés les constituants même du projet. Nous insisterons donc sur les points de vue organisationnels liés à la conception de notre solution technique.Après avoir resitué le contexte du problème, il s'agira de définir les objectifs ainsi que les contraintes liés à ce type de projet. Sont détaillées aussi dans ce document, les méthodes de travail,la répartition des rôles au sein du groupe d'ingénierie ainsi que le planning de répartition des tâches et des charges de travail. Une description précise des livrables sera effectuée et nous donnerons un aperçu des risques que nous sommes susceptible de rencontrer.

    \chapter{Contexte du document}
Dans le cadre d'un projet d'ingénierie, l'utilité d'un dossier d'initialisation est double. Tout d'abord, celui-ci permet de préciser au client quelle seront les modalités formelles des documents et des livrables qui lui seront fournis. Ce dossier permet de lancer le projet en ce qu'il pose les bases et détaille le contenu de ce que le groupe d'ingénierie va produire tout au long des phases de vie d'un projet (spécification, conception, réalisation, tests, intégration, recette…).D'autre part, ce dossier d'initialisation sert de référentiel commun au groupe de travail. Il permet de clarifier les procédures de travail au sein du groupe humain. Ainsi s'il arrive qu'un acteur s'éloigne de ses objectifs principaux, il suffira de se référer au dossier d'initialisation pour recadrer son travail. Le Macro-Phasage permet de même de prévoir l'évolution du travail et de constater les éventuels écarts de entre ce qui avait été prévu et ce qui a été réellement fait, c'est donc un outil de gestion de risque.
    \chapter{Documents de référence}
%que mettre dans cette partie->voir avec Jan.
    
    \chapter{Rappel du problème}
Le COPEVUE a lancé un appel d'offre dans le cadre de la réalisation d'un système de monitoring de sites isolés, il s'agit donc de concevoir en premier lieu une solution technique permettant de répondre aux mieux aux exigences fonctionnelles et non fonctionnelles que le COPEVUE formule. De façon synthétique notre équipe va proposer une solution permettant de surveiller des sites naturels difficiles d'accès (souvent à cause des conditions environnementales) et peu peuplés. 
    \section{Le contexte}
    Voici un peu de texte de section.
    \section{Les objectifs}
    Voici un peu de texte de section.
    
    \chapter{Les contraintes générales}
    Voici un peu de texte.
    \section{Confidentialité}
    Voici un peu de texte de section.
    \section{Etude de l'existant}
    Voici un peu de texte de section.
    \section{Exigences non fonctionnelles}
    Voici un peu de texte de section.
    
    \chapter{Organisation du travail}
    Voici un peu de texte.
    \section{Chef de projet (CdP)}
    Voici un peu de texte de section.
    \section{Responsable Qualité, Méthode et Documentation}
    Voici un peu de texte de section.
    \section{Groupe d'études informatique}
    Voici un peu de texte de section.
    
    \chapter{Listes des livrables attendus}
    Voici un peu de texte.
    \section{Chef de projet}
    Voici un peu de texte de section.
    \section{Responsable Qualité}
    Voici un peu de texte de section.
    \section{Groupe d'études informatique}
    Voici un peu de texte de section.
    
    \chapter{Organigramme des tâches}
    Voici un peu de texte.
    \section{Macro-Phasage}
    Voici un peu de texte de section.
    \section{Diagramme de Gantt}
    Voici un peu de texte de section.
    
    \chapter{Modalités de suivi}
    Voici un peu de texte.
    \section{Les règles de suivi}
    Voici un peu de texte de section.
    \section{Les outils utilisés}
    Voici un peu de texte de section.
    \section{Procédures de révisions du planning}
    Voici un peu de texte de section.
    
    \chapter{Gestion des risques}
    Voici un peu de texte.
    \section{Risques concernant l'application du projet}
    Voici un peu de texte de section.
    \section{Risques propres au projet}
    Voici un peu de texte de section.
    
    \chapter{Conclusion}
    Voici un peu de texte.
    
    \chapter{Annexes}
    Voici un peu de texte.
    \section{Les fiches de tâche}
    Voici un peu de texte de section.
    \section{Exemple de fiche de revue}
    Voici un peu de texte de section.
    \section{Exemple de fiche de séance}
    Voici un peu de texte de section.



\end{document}