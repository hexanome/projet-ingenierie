\section{Relevé de décisions}
Dans le cadre de la réalisation de ce projet notre groupe de travail a souvent été amené à faire des choix, que ce soit des choix techniques ou des choix plus organisationnels. Certains choix importants sont ainsi précisés dans ce document.
\subsection{Cas d'étude environnemental}
Après avoir pris connaissance de la demande du COPEVUE, la première chose à faire était de positionner la solution technique que nous allions développer en trouvant un site de référence qui allait représenter "le pire cas" de site isolé à monitorer. Notre groupe de travail est donc tombé d'accord sur le fait de travailler sur un site nordique ou le soleil n'est que peu présent pendant les longues périodes d'hiver et ou les températures sont parfois très froides (jusqu'a -40°C). Melle Tyrvainen étant une étudiante étrangère finlandaise, son aide nous a été précieuse pour établir les caractéristiques de l'environnement d'évolution de notre réseau de capteurs.
\medskip
\subsection{Production d'énergie}
Nous avons initié notre travail par la partie concernant la production d'énergie au sein de la station d'acquisition des données capteurs sur le site à monitorer. Le choix que nous avons fait aura donc été d'utiliser la pile à hydrogène pour la production d'énergie sur site. L'avantage principal de cette ressource étant de pouvoir l'adapter dans de nombreux endroits à travers le monde indépendemment des conditions climatiques (Ce qui n'était pas le cas avec le solaire ou l'éolien).De même, la pile à combustible demande relativement peu d'entretien et permet donc d'espacer les interventions humaines sur site.
\medskip
\subsection{Acquisition des données}
En ce qui concerne l'acquisition des données, nous avons déterminés que le fait de relever les valeurs des capteurs 4 fois par 24h était suffisant. Nous avons de même admis que l'envoi de ces informations se ferait à chaque relevé de valeurs afin de ne pas multiplier les phases d'arrêt/démarrage très consomatrices en énergie.
\medskip
\subsection{Transmission des données}
Pour la transmission des données de la station d'accueil vers le centre de supervision nous avons choisi de pouvoir transférer les données à la fois par GPS et/ou par GPRS, ceci permet donc de pouvoir communiquer sur toutes les zones du globes. Pour la solution mettant en jeu la technologie GPS il est possible que celle-ci implique un surcoût du fait qu'elle soit plus compliquée à mettre en oeuvre. La transmission des données se fait de manière générale par le biais de requètes http via un serveur web accessible à distance.
\medskip
\subsection{Restitution des données}
Notre choix pour la restitution des données aura été de faire une application complétement délocalisée sur internet et accessible via un navigateur. L'utilité de ce genre de solution est sa facilité et sa rapdidité d'accès. Ainsi le monitoring pourra être éffectué partout autour du monde. De même, il nous a semblé intéressant de fournir une interface à l'intervenant qui se rend sur place pour maintenir l'installation de capteurs ou encore les réservoirs en place c'est pourquoi les données traitées via le serveur central seront adaptable à un affichage sur un smartphone.