\section{Introduction}

\subsection{Présentation du projet}

% On reprend ce qui a déjà été fait dans les autres documents.

\subsection{Présentation du document}

Nous allons dans ce cahier des charges présenter en détails les besoins que... (presentation information et servises web)

\subsection{Documents applicables / Documents de référence}

% Décider ensemble de ce que l'on met dans ce genre de sections.

\subsection{Terminologie et abréviations}

% A complêter au fur et à mesure.
% Définir le "Système de gestion du site isolé".
% Définir le "Système central".

\section{Présentation du problème}

\subsection{Buts, nature du logiciel, utilisateurs concernés}


\subsection{Formulation des besoins, exploitation et ergonomie, expérience}


\subsection{Portée, développement, mise en oeuvre, organisation de la maintenance}



\subsection{Limites}

\section{Exigences fonctionnelles}
\subsection{Fonctions de base, performances et aptitudes}
\subsection{Contraintes d'utilisation}
\subsection{Critères d'appréciation de la réalisation effective de la fonction}
\subsection{Flexibilité dans la façon de mettre en oeuvre la fonction convernée, variation de coûts associée en fonciton de cette flexibilité}

\section{Exigences non fonctionnelles}

\section{Contraintes imposées, faisabilité technologique, moyens}
\subsection{Sûreté, planning, organinsation, communication}
\subsection{Complexité}
\subsection{Compétenes, moyens et règles}
\subsection{Normes de documentation}

\section{Configuration cible}
\subsection{Matériels et logiciels}
\subsection{Stabilité de la configuration}
\subsection{Description des API}

\section{Guide de réponse au cahier des charges}
\subsection{Grille d'évaluation}

\section{Annexes}
\subsection{Observations de l'existant}
\subsection{Propositions d'orientation}
\subsection{Images d'écrans principaux du logiciel}
\subsection{Résultat de l'analyse de la valeur}