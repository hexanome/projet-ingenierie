\section{Introduction}

\subsection{Présentation du projet}

% On reprend ce qui a déjà été fait dans les autres documents.

\subsection{Présentation du document}

Ce dossier présentent les besoins que nous avons par rapport à la solution d'acquisition des données des capteurs, et de la transmission de celles-ci avec la base de la station. Nous essayerons tout d'abord de voir quelles sont les objectifs de cette solution technique, les utilisateurs concernés et les problèmes pouvant se présenter, autant dans son développement, sa mise en oeuvre, et finalement sa maintenance. Ce document est accompagné d'annexes permettant d'appronfondir certains sujets plus techniques sur lesquels nous n'avons pas voulu nous attarder dans un but de clarté.

\subsection{Documents applicables / Documents de référence}

% Décider ensemble de ce que l'on met dans ce genre de sections.

\subsection{Terminologie et abréviations}

% A complêter au fur et à mesure.
\begin(itemize)
\item Base (de la station) : système informatique équipé d'un OS linux permettant de communiquer avec les capteurs et le serveur central. La base peut aussi enregistrer quelques données et permet de configurer la plupart des éléments de la station.
\item Mode sommeil : mode dans lequel le microcontroleur et le capteur ne consomme que très peu d'énergie. Aucune acquisition et aucun calcul ne doivent être fait pendant cette periode. Le micro-controleur attend d'être réveillé par un évenement materiel ou logiciel.
\end(itemize)

\section{Présentation du problème}

\subsection{Buts, nature du logiciel, utilisateurs concernés}

Le logicel qui doit être implementé a trois fonctions:
\begin{itemize}
\item L'acquisition des données des capteurs à un intervalle de temps définie préalablement,
\item L'envoi de ses données à la base de la station à un intervalle de temps régulier,
\item La configuration des capteurs par la base. Ainsi, l'equart entre chaque acquisition ou transismion devra être parametrable. Cette configuration sera faite pendant la transmission avec la base.
\end{itemize}

Ce composant logiciel devra ainsi intéragir avec la base de la station, lui envoyé les données et vérifier si une reconfiguration est necessaire. Il devra également se charger de récuperer la valeur du capteur auquel il est attaché. Ce composant logiciel devra impérativement être économes et devra passer la grande majorité du temps en mode "sommeil".

D'un point de vue plus technique, ce logiciel devra s'exécuter sur un micro-controleur attaché à une puce ZigBee et au capteur.

Le protocole de communication utilisé par les modules de transmission avec la base sera décrit dans ce document.

Ce document va aussi porter sur le matériel exploité par le logiciel précédement cité, à savoir le module de communication ZigBee. Les attentes par rapport à celui-ci seront décrite plus loins.

\subsection{Formulation des besoins, exploitation et ergonomie, expérience}

Le système présentement présenté doit répondre à deux besoins bien distincts, et ce quelque soit la localisation géographique du système dans le monde :

\parapraph{Acquisition} Le système doit pouvoir recuperer la valeur des capteurs de la façon suivante :

\begin{itemize}
\item Le micro-controlleur sera reveillé selon la periode d'acquisition, il relevera la valeur du capteur 
\item Selon la configuration de la periode de transmission, soit la donnée du capteur est en enregistré temporairement, soit directement envoyé à la base
\item Le micro-controlleur devra s'adapter au capteur pour pouvoir gerer les differents types d'acquisitions
\end{itemize}

\paragraph{Transmission} Après avoir relevé les données, le système doit pouvoir envoyé les informations à la base :

\begin{itemize}
\item Les différents quantum de temps définissant la granularité avec laquelle le système transmet les résultats a la base, en analysant les données des capteurs.
\item Le protocole d'envoi des données décrit plus loin devra être respecté. 
\end{itemize}

\subsection{Portée, développement, mise en oeuvre, organisation de la maintenance}


\subsection{Limites}

\section{Exigences fonctionnelles}
\subsection{Fonctions de base, performances et aptitudes}
\subsection{Contraintes d'utilisation}

Le système ne doit pas pouvoir être utilisé par un utilisateur humain. La sécurisation des données envoyées est à prévoir, il ne faut pas qu'un utilisateur externe puissent interagir avec le systeme. Cette interaction pourrait mener à des disfonctionnements du système. Un système de cryptage, tel qu'un certificat, ou une autre méthode de sécurisation doit être mis en place.

\subsection{Critères d'appréciation de la réalisation effective de la fonction}
\subsection{Flexibilité dans la façon de mettre en oeuvre la fonction convernée, variation de coûts associée en fonciton de cette flexibilité}

\section{Exigences non fonctionnelles}

\section{Contraintes imposées, faisabilité technologique, moyens}
\subsection{Sûreté, planning, organinsation, communication}
\subsection{Complexité}
\subsection{Compétenes, moyens et règles}
\subsection{Normes de documentation}

\section{Configuration cible}
\subsection{Matériels et logiciels}
\subsection{Stabilité de la configuration}
\subsection{Description des API}

\section{Guide de réponse au cahier des charges}
\subsection{Grille d'évaluation}

\section{Annexes}
\subsection{Observations de l'existant}
\subsection{Propositions d'orientation}
\subsection{Images d'écrans principaux du logiciel}
\subsection{Résultat de l'analyse de la valeur}