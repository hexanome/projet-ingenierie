%-------------------------------------------
% En-tête type de document pour le projet PLD
% Il suffit de remplir le input ligne 45
%-------------------------------------------

\documentclass[a4paper]{article}

\usepackage[utf8]{inputenc}   
\usepackage[top=2cm, bottom=2cm, left=2cm, right=2cm]{geometry}
\usepackage{ucs}
% Reconnaitre les caratères accentués dans le source.
\usepackage[T1]{fontenc} 
\usepackage{lmodern}
\usepackage[francais]{babel}
% Insertion d'images
\usepackage{graphicx}
% Utilisation du symbole EURO
\usepackage{eurosym}

\setlength{\parskip}{10pt plus 1pt minus 1pt}

\begin{document}

%------------------------------------- Page de titre
\begin{titlepage}
~ 
\vfill
	\begin{center}
		\begin{Huge}
		Cahier des charges :\\ Logiciel de communication base - centrale
		\end{Huge} 
\vfill
		\textbf{Hexanome 4111 :} 
		\\Quentin \bsc{Calvez}, Matthieu \bsc{Coquet}, 
		\\Jan \bsc{Keromnes}, Alexandre \bsc{Lefoulon}, 
		\\Thaddée \bsc{Tyl}, Xavier \bsc{Sauvagnat},
		\\Tuuli \bsc{Tyrvainen}\\
\vfill		
		\begin{Large}
		Janvier 2012
		\end{Large}
\vfill
	\begin{tabular}{|c|c|c|c|}
 	 \hline
	\end{tabular}
	\end{center}
\vfill
\end{titlepage}
%----------------------------------------------------
%--------------------------------- Table des matières
\newpage
\tableofcontents
\newpage
%----------------------------------------------------

%------------------- Insertion du contenu du document
\section{Introduction}

\subsection{Présentation du projet}

% On reprend ce qui a déjà été fait dans les autres documents.

\subsection{Présentation du document}

Nous allons dans ce cahier des charges présenter en détails les besoins que... (presentation information et servises web)

\subsection{Documents applicables / Documents de référence}

% Décider ensemble de ce que l'on met dans ce genre de sections.

\subsection{Terminologie et abréviations}

% A complêter au fur et à mesure.
% Définir le "Système de gestion du site isolé".
% Définir le "Système central".

\section{Présentation du problème}

\subsection{Buts, nature du logiciel, utilisateurs concernés}

% Tuuli: l’interface entre utilisateur et système → entrée doit être facile pour donner et sortie doit etre facile pour comprendre → en général prend en compte utilisabilité → minimise les erreurs et assure le usage efficace


\subsection{Formulation des besoins, exploitation et ergonomie, expérience}


\subsection{Portée, développement, mise en oeuvre, organisation de la maintenance}



\subsection{Limites}

\section{Exigences fonctionnelles}
\subsection{Fonctions de base, performances et aptitudes}
\subsection{Contraintes d'utilisation}
\subsection{Critères d'appréciation de la réalisation effective de la fonction}
\subsection{Flexibilité dans la façon de mettre en oeuvre la fonction convernée, variation de coûts associée en fonciton de cette flexibilité}

\section{Exigences non fonctionnelles}

\section{Contraintes imposées, faisabilité technologique, moyens}
\subsection{Sûreté, planning, organinsation, communication}
\subsection{Complexité}
\subsection{Compétenes, moyens et règles}
\subsection{Normes de documentation}

\section{Configuration cible}
\subsection{Matériels et logiciels}
\subsection{Stabilité de la configuration}
\subsection{Description des API}

\section{Guide de réponse au cahier des charges}
\subsection{Grille d'évaluation}

\section{Annexes}
\subsection{Observations de l'existant}
\subsection{Propositions d'orientation}
\subsection{Images d'écrans principaux du logiciel}
\subsection{Résultat de l'analyse de la valeur}
%----------------------------------------------------

%------------------- 

%----------------------------------------------------

\end{document}