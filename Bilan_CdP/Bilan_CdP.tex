Ce projet est pour notre hexanome assez nouveau, il s'inscrit cependant dans l'évolution de notre cursus d'ingénieur informatique. Nous prenons conscience que les spécifications techniques des produits sont indispensables lorsque l'on veut déployer une solution à grande échelle. Alors que l'informatique s'affranchit souvent des contraintes de distances, de météo ou d'environnement nous nous sommes replacé ici dans un contexte plus industriel qui imposait que notre solution respecte de fortes contraintes d'autonomie énergétique et logicielles.Bien que le domaine technologique soit différent, ce projet était assez lié avec le PLD SI ce qui pouvait parfois décourager les membres du GEI devant la quantité assez conséquente de documents à rédiger. L'intérèt de ce projet aura été sans nul doute de poser sur le papiers les moyens à mettre en place pour réaliser un système de monitoring complet. Nous prenons en considération le fait que chaque brique matérielle ou logicielle doit être parfaitement spécifiée si l'on veut palier aux problème mais aussi facilité la réalisation ou le déploiement. C'était pour ainsi dire la première fois qu'un projet si important nous était confié et la partie organisationnelle est conséquente à gérer pour le chef de projet, notamment dans la synchronisation des équipes de travail. En effet, il était nécessaire que certains documents soient réalisés en priorité par le responsable qualité afin que ceux-ci puissent servir pour la rédaction de certains documents chez les GEI.